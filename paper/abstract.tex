The strength of a dynamic language is also its weakness: run-time
flexibility comes at the cost of compile-time predictability.
%
Many of the hallmarks of dynamic languages such as closures,
continuations, various forms of reflection, and a lack of static types
make many programmers rejoice, while compiler writers, tool
developers, and verification engineers lament.
%
The dynamism of these features simply confounds statically reasoning
about programs that use them.
%
Consequently, static analyses for dynamic languages are few, far
between, and seldom sound.

The ``abstracting abstract machines'' (AAM) approach to constructing
static analyses has recently been proposed as a method to ameliorate
the difficulty of designing analyses for such language features.
%
The approach, so called because it derives a function for the sound
and computable approximation of program behavior starting from the
abstract machine semantics of a language, provides a viable approach
to dynamic language analysis since all that is required is a machine
description of the interpreter.


The AAM recipe as originally described produces finite state
abstractions: the behavior of a program is approximated as a finite
state machine.
%
Such a model is inherently imprecise when it comes to reasoning about
the control stack of the interpreter: a finite state machine cannot
faithfully represent a stack.
%
Recent advances have shown that higher-order programs can be
approximated with pushdown systems.
%
However, such models, founded in automata theory, either breakdown or
require significant engineering in the face of dynamic language
features that inspect or modify the control stack.

In this paper, we tackle the problem of bringing pushdown flow
analysis to the domain of dynamic language features.  We revise the
abstracting abstract machines technique to target the stronger
computational model of pushdown systems.
%
In place of automata theory, we use only abstract machines and
memoization.
%
As case studies, we show the technique applies to a language with
closures, garbage collection, stack-inspection, and first-class
composable continuations.
