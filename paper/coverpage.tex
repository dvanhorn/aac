\begin{titlepage}
  \begin{center}
    \Huge{Issues raised by reviewers:}
  \end{center}
  \begin{itemize}
  \item{\textbf{Contribution magnitude poorly motivated}
      We toned down the grandiose language and discussed problem domains that benefit.}
  \item{\textbf{Poorly explained how technique works and what end results are}
      We've taken more space and care in our wording to explain the concepts.
      We added an additional example program that illustrates the weakness of \zcfa{} and strength of pushdown abstraction.
%
      We gave graphical visualizations of the otherwise mystical Greek.
%
    Results beyond the exemplified programs' improved control flow predictions are not included for a few reasons:
    \begin{itemize}
    \item{the paper derives existing analyses that have already been evaluated by the associated literature}
    \item{lack of corpus for CM use within a language less complex than Java or Racket meant a lot more work outside our existing analysis implementation for a small Scheme.
%
Interesting direction, but too much for one paper.}
    \item{the implementation at \url{https://github.com/dvanhorn/oaam/tree/aac} is mostly within an order of magnitude (slower) than the baseline regular analysis (a few seconds instead of a fraction of a second, in most cases), based on informal benchmarking.
        %
We experimented with non-traditional binding strategies and got more variance.
%
Ultimately, we ran out of time to work the lead out to get a fully comparative evaluation like in our Optimizing Abstract Abstract Machines paper.
%
It's a lesser paper for it, admittedly, but still holds its water.}
    \end{itemize}
  }
  \item{\textbf{Too formal for DLS}
      Theorems and proofs have been moved to an extended version to appear on \url{arXiv.org}, self-titled.
      We added a new section post-introduction to ease into the details.
      Word choices are livelier.}
  \end{itemize}
\end{titlepage}
