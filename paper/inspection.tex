Since we just showed how to produce a pushdown system from an abstract machine, some readers may be concerned that we have lost the ability to reason about the stack as a whole.
%
This is not the case.
%
The semantics may still refer to $\mktab$ to make judgments about the possible stacks that can be realized at each state.
%
The key is to interpret the functions making these judgments again with the AAM methodology.

%
Some semantic features allow a language to inspect some arbitrarily deep part of the stack, or compute a property of the whole stack before continuing.
%
Java's access control security features are an example of the first form of inspection, and garbage collection is an example of the second.
%
We will demonstrate both forms are simple first-order metafunctions that the AAM methodology will soundly interpret.
%
Access control can be modeled with continuation marks, so we demonstrate with the CM machine of \citeauthor{dvanhorn:Clements2004Tailrecursive}.

%%
Semantics that inspect the stack do so with metafunction calls that recur down the stack.
%
Recursive metafunctions can be thought of as out-of-band reduction relations that match rules in order until one fires, repeatedly until a return value is produced.
%
Interpreted via AAM, non-deterministic metafunction evaluation leads to a set of possible results.
%
The finite restriction on the state space carries over to metafunction inputs, so we can always detect infinite loops and bail out of that execution path.
%
Specifically, a metafunction call can be seen as an initial state, $s$, in a reduction system for which we can compute all terminal states:
\begin{align*}
  \terminal &: \forall A. (\text{relation } A)^* \times A \to \wp(A) \\
  \terminal(\mathit{Rs},s) &= \terminalaux(\set{s},\set{s},\emptyset) \\[2pt]
  \text{where } \terminalaux(S, \emptyset, T) &= T \\
   \terminalaux(S, F \uplus \set{s}, T) &= \terminalaux(S\cup I, F \cup (I \setminus S), T') \\
   & \text{where } (I',T') = \trystep(\mathit{Rs}) \\
   \trystep(\epsilon) &= (\emptyset,T\cup\set{s}) \\
   \trystep(\stepto:\mathit{Rs}) &= I\deceq\emptyset \to \trystep(\mathit{Rs}), (I, T) \\
    & \text{where } I = \setbuild{s'}{s \stepto s'}
\end{align*}

This definition is a typical worklist algorithm.
%
It builds the set of terminal terms, $T$, by exploring the frontier (or worklist), $F$, and only adding terms to the frontier that have not been seen, as represented by $S$.
%
If a rule does not match a given term, the intermediate set, $I$ will be empty, and $\terminalaux$ will continue trying the following rules until it finds a match.
%
It is possible for metafunctions' rewrite rules to themselves use metafunctions, but the \emph{seen} set for $\terminal$ must be dynamically bound\footnote{This is a reference to dynamic scope as opposed to lexical scope.} -- it cannot restart at $\emptyset$ upon reentry.
%
Without this precaution, the host language will exceed its stack limits when an infinite path is explored, rather than bail out.

\subsection{Case study for stack traversal: GC}
Garbage collection is an example of a language feature that needs to crawl the stack, specifically to find live addresses.
%
We are interested in garbage collection because it can give massive precision boosts to analyses~\citep{dvanhorn:Might:2006:GammaCFA,dvanhorn:Earl2012Introspective}.
%
The following function will produce the set of live addresses in the stack:

\begin{align*}
  \kontlive &: \Frame^* \to \wp(\Addr) \\
  \kontlive(\makont) &= \kontliveaux(\makont,\emptyset) \\[2pt]
  \kontliveaux(\epsilon,L) &= L \\
  \kontliveaux(\kcons{\mkframe}{\makont}, L) &= \kontliveaux(\makont, L\cup\touches(\mkframe)) \\
  \text{where } \touches(\appl{\mexpr,\menv}) &= \touches(\appr{\mexpr,\menv}) = \touches(\mexpr,\menv) \\
                \touches(\mexpr,\menv) &= \setbuild{\menv(\mvar)}{\mvar \in \fv(\mexpr)}
\end{align*}

When interpreted via AAM, the continuation is indirected through $\mktab$ and leads to multiple results, and possibly loops through $\mktab$.
%
Thus this is more properly understood as
\begin{align*}
  \kontlive(\mktab,\makont) &= \terminal(\set{\stepto_i}_{i=0}^1, \kontliveaux(\mktab,\makont,\emptyset)) \\[2pt]
  \kontliveaux(\mktab,\epsilon,L) &\stepto_0 L \\
  \kontliveaux(\mktab,\kcons{\mkframe}{\mctx}, L) &\stepto_1 \kontliveaux(\mktab,\makont, L\cup\touches(\mkframe)) \text{ if } \makont \in \mktab(\mctx)
\end{align*}
%

A garbage collecting semantics can choose to collect the heap with respect to each live set (call this $\Gamma^*$), or, soundly, collect with respect to their union (call this $\hat\Gamma$).\footnote{The garbage collecting version of PDCFA~\citep{ianjohnson:DBLP:journals/jfp/JohnsonSEMH14} evaluates the $\hat\Gamma$ strategy.}
%
On the one hand we could have tighter collections but more possible states, and on the other hand we can leave some precision behind in the hopes that the state space will be smaller.
%
In the general idea of relevance versus irrelevance, the continuation's live addresses are relevant to execution, but are already implicitly represented in contexts because they must be mapped in the store's domain.

%
A state is ``collected'' only if live addresses remain in the domain of $\mstore$.
%
We say a value $\mval \in \mstore(\maddr)$ is live if $\maddr$ is live.
%
If a value is live, any addresses it touches are live; this is captured by the computation in $\reaches$:
%
\begin{align*}
  \reaches(\mathit{root},\mstore) &=
 \setbuild{\maddralt}{\maddr \in \mathit{root}, \maddr \leadsto_\mstore^* \maddralt} \\
&  \infer{\mval \in \mstore(\maddr) \\ \maddralt \in \touches(\mval)}{\maddr \leadsto_\mstore \maddralt}
\end{align*}
So the two collection methods are as follows.
%
Exact GC produces different collected states based on the possible stacks' live addresses:\footnote{It is possible and more efficient to build the stack's live addresses piecemeal as an additional component of each state, precluding the need for $\kontlive$. Each stack in $\mktab$ would also store the live addresses to restore on pop.}
\begin{align*}
  \Gamma^*(\mastate,\mktab) &=
    \setbuild{\mastate\set{\mstore:=\mastate.\mstore|_L}}{L \in \live^*(\mastate,\mktab)} \\
  \live^*(\tpl{\mexpr,\menv,\mstore,\makont},\mktab) &=
    \setbuild{\reaches(\touches(\mexpr,\menv) \cup L, \mstore)}{L \in \kontlive(\mktab,\makont)}
\end{align*}
\begin{equation*}
  \infer{\mastate,\mktab \stepto \mastate',\mktab' \\
         \mastate' \in \Gamma^*(\mstate',\mktab')}
        {\mastate,\mktab \stepto_{\Gamma^*} \mastate',\mktab}
\end{equation*}
And inexact GC produces a single state that collects based on all (known) stacks' live addresses:
\begin{align*}
  \hat\Gamma(\mastate,\mktab) &=
  \mastate\set{\mstore:=\mastate.\mstore|_{\widehat{\live}(\mastate,\mktab)}} \\
  \widehat{\live}(\tpl{\mexpr,\menv,\mstore,\makont},\mktab) &=
    \reaches(\touches(\mexpr,\menv) \cup \bigcup\kontlive(\mktab,\makont), \mstore)
\end{align*}
\begin{equation*}
  \infer{\mastate,\mktab \stepto \mastate',\mktab'}
        {\mastate,\mktab \stepto_{\hat\Gamma} \hat\Gamma(\mstate',\mktab')\mktab'}
\end{equation*}

Without the continuation store, the baseline GC is
\begin{align*}
  \Gamma(\mastate) &= \mastate\set{\mstore:=\mastate.\mstore|_{\live(\mastate)}} \\
  \live(\mexpr,\menv,\mstore,\mkont) &= \reaches(\touches(\mexpr,\menv)\cup \kontlive(\makont), \mstore)
\end{align*}
\begin{equation*}
  \infer{\mastate \stepto \mastate'}
        {\mastate \stepto_{\Gamma} \Gamma(\mstate')}  
\end{equation*}
Suppose at arbitrary times we decide to perform garbage collection rather than continue with garbage.
%
So when $\mastate \stepto \mastate'$, we instead do $\mastate \stepto_\Gamma \mastate'$.
%
The times we perform GC do not matter for soundness, since we are not analyzing GC behavior.
%
However, garbage stands in the way of completeness.
%
Mismatches in the GC application for the different semantics lead to mismatches in resulting state spaces, not just up to garbage in stores, but in spurious paths from dereferencing a reallocated address that was not first collected.
%

%
The state space compaction that continuation stores give us makes ensuring GC times match up for the completeness proposition tedious.
%
Our statement of completeness then will assume both semantics perform garbage collection on every step.
%

\begin{lemma}[Correctness of $\kontlive$]
  For all $\mktab,\mkont,\makont$,
  \begin{itemize}
  \item{\textbf{Soundness:} if $\mkont \in \unroll{\mktab}{\makont}$ then $\kontlive(\mkont) \in \kontlive(\mktab,\mkont)$}
  \item{\textbf{Completeness:} for all $L \in \kontlive(\mktab,\makont)$ there is a $\mkont \in \unroll{\mktab}{\makont}$ such that $L = \kontlive(\mkont)$.}
  \end{itemize}
\end{lemma}
\begin{proof}
  Soundness follows by induction on the unrolling. Completeness follows by \todo{Ian}.
\end{proof}
\begin{theorem}[Correctness of $\Gamma^*\CESKKstart$]
  For all expressions $\mexpr_0$,
  \begin{itemize}
  \item{{\bf Soundness: } %for all $\mstate\equiv\tpl{\mexpr,\menv,\mkont},\mstate'\equiv{\mexpr',\menv',\mkont'} \in \CESKt$,
        if $\mstate \stepto_{\Gamma\CESKt} \mstate'$,
        %for all $\mktab,\makont$ if
        $\inv(\mstate\set{\mkont := \makont},\mktab)$,
        and $\mstate.\mkont \in \unroll{\mktab}{\makont}$, then
        there are $\mktab',\makont',\mstore'$ such that
        $\mstate\set{\mkont := \makont},\mktab \stepto_{\Gamma^*\CESKKstart} \mastate',\mktab'$ where
        $\mastate' = \mstate'\set{\mkont := \makont',\mstore:=\mstore'}$ 
        and $\mstate'.\mkont \in \unroll{\mktab'}{\makont'}$
        and finally there is an $L \in \live^*(\mastate',\mktab')$ such that $\mstore'|_L = \mstate'.\mstore|_{\live(\mstate')}$}
  \item{{\bf Completeness:} if $\mastate\equiv\tpl{\mexpr,\menv,\mstore,\makont},\mktab \stepto_{\Gamma^*\CESKKstart} \mastate',\mktab'$ and there is an $L_\mkont \in \kontlive(\mktab,\makont)$ such that $\mstore|_L = \mstore$ (where $L = \reaches(\touches(\mexpr,\menv) \cup L_\mkont, \mstore)$) and $\inv(\mastate,\mktab)$,
      for all $\mkont \in \unroll{\mktab}{\makont}$ such that $\kontlive(\mkont) = L_\mkont$,
      there is a $\mkont'$ such that
      $\mastate\set{\makont := \mkont} \stepto_{\Gamma\CESKt}
      \mastate'\set{\makont := \mkont'}$ (a GC step) and
      $\mkont' \in \unroll{\mktab}{\mastate'.\makont}$}
  \end{itemize}  
\end{theorem}

\begin{theorem}[Soundness of $\hat\Gamma\CESKKstart$]
  For all expressions $\mexpr_0$,
  if $\mstate \stepto_{\Gamma\CESKt} \mstate'$,
  % for all $\mktab,\makont$ if
  $\inv(\mstate\set{\mkont := \makont},\mktab)$,
  and $\mstate.\mkont \in \unroll{\mktab}{\makont}$, then
  there are $\mktab',\makont',\mstore''$ such that
  $\mstate\set{\mkont := \makont},\mktab \stepto_{\hat\Gamma\CESKKstart} \mastate',\mktab'$ where
  $\mastate' = \mstate'\set{\mkont := \makont',\mstore:=\mstore''}$
  and $\mstate'.\mkont \in \unroll{\mktab'}{\makont'}$
  and finally $\mstate'.\mstore|_{\live(\mstate')} \sqsubseteq \mstore''|_{\widehat{live}(\mastate',\mktab')}$
\end{theorem}
The proofs are straightforward, and use the usual lemmas for GC, such as idempotence of $\Gamma$ and $\touches \subseteq \reaches$.
\subsection{Case study analyzing security features: the CM machine}
The CM machine provides a model of access control: a dynamic branch of execution is given permission to use some resource if a continuation mark for that permission is set to ``grant.''
%
There are three new forms we add to the lambda calculus to model this feature: {\tt grant}, {\tt frame}, and {\tt test}.
%
The {\tt grant} construct adds a permission to the stack.
%
The concern of unforgeable permissions is orthogonal, so we simplify with a set of permissions that is textually present in the program:
\begin{align*}
  \mperm \in \Permissions & \text{ a set} \\
  \Expr &::= \ldots \alt \sgrant{\mperm}{\mexpr} \alt \sframe{\mperm}{\mexpr} \alt \stest{\mperm}{\mexpr}{\mexpr}
\end{align*}
%
The {\tt frame} construct ensures that only a given set of permissions are allowed, but not necessarily granted.
%
The security is in the semantics of {\tt test}: we can test if all marks in some set $\mperm$ have been granted in the stack without first being denied; this involves crawling the stack:
\begin{align*}
  \OK(\emptyset,\mkont) &= \mathit{True} \\
  \OK(\mperm,\epsilon^\mpermmap) &= \passp(\mperm,\mpermmap) \\
  \OK(\mperm,\kconsm{\mkframe}{\mpermmap}{\mkont}) &= \passp(\mperm,\mpermmap) \textbf{ and } \OK(\mperm \setminus \mpermmap^{-1}(\Grant), \mkont) \\
  \text{where }\passp(\mperm,\mpermmap) &= \mperm \cap \mpermmap^{-1}(\Deny) \deceq \emptyset
\end{align*}
The set subtraction is to say that granted permissions do not need to be checked farther down the stack.
%

%
Continuation marks respect tail calls and have an interface that abstracts over the stack implementation.
%
Each stack frame added to the continuation carries the permission map.
%
The empty continuation also carries a permission map.
%
Crucially, the added constructs do not add frames to the stack; instead, they update the permission map in the top frame, or if empty, the empty continuation's permission map.
\begin{align*}
  \mpermmap \in \PermissionMap &= \Permissions \finto \GD \\
  \mgd \in \GD &::= \Grant \alt \Deny \\
  \mkont \in \Kont &::= \epsilon^\mpermmap \alt \kconsm{\mkframe}{\mpermmap}{\mkont}
\end{align*}
Update for continuation marks:
\begin{align*}
  \extm{\mpermmap}{\mperm}{\mgd} &= \lambda x. x \decin \mperm \to \mgd, \mpermmap(x) \\
  \extm{\mpermmap}{\overline{\mperm}}{\mgd} &= \lambda x. x \decin \mperm \to \mpermmap(x),\mgd \end{align*}

\begin{figure}
  \centering
  \begin{tabular}{r|l}
    \hline\vspace{-3mm}\\
    $\tpl{\sgrant{\mperm}{\mexpr}, \menv, \mstore, \mkont}$
    &
    $\tpl{\mexpr,\menv,\mstore, \extm{\mkont}{\mperm}{\Grant}}$
    \\
    $\tpl{\sframe{\mperm}{\mexpr}, \menv,\mstore,  \mkont}$
    &
    $\tpl{\mexpr,\menv,\mstore, \extm{\mkont}{\overline{\mperm}}{\Deny}}$
    \\
    $\tpl{\stest{\mperm}{\mexpri0}{\mexpri1}, \menv,\mstore,  \mkont}$
    &
    $\tpl{\mexpri0,\menv,\mstore, \mkont}$ if $\mathit{True} = \OK(\mperm,\mkont)$
    \\
    &
    $\tpl{\mexpri1,\menv,\mstore, \mkont}$ if $\mathit{False} = \OK(\mperm,\mkont)$
  \end{tabular}
  \caption{CM machine semantics}
  \label{fig:cm-semantics}
\end{figure}

The abstract version of the semantics has one change on top of the usual continuation store.
%
The {\tt test} rules are now
\begin{align*}
  \tpl{\stest{\mperm}{\mexpri0}{\mexpri1}, \menv, \mstore, \makont},\mktab
  &\stepto
  \tpl{\mexpri0,\menv,\mstore, \makont},\mktab \text{ if } \mathit{True} \in \widehat{\OK}(\mktab,\mperm,\makont)
  \\
  &\stepto
  \tpl{\mexpri1,\menv,\mstore,\makont},\mktab \text{ if } \mathit{False} \in \widehat{\OK}(\mktab,\mperm,\makont)
\end{align*}
where the a new $\widehat{\OK}$ function uses $\terminal$ and rewrite rules:
\begin{align*}
  \widehat{\OK}(\mktab,\mperm,\makont) &= \terminal(\set{\stepto_i}_{i=0}^2,\widehat{\OK}^*(\mktab,\mperm,\makont)) \\[2pt]
  \widehat{\OK}^*(\mktab,\emptyset,\makont) &\stepto_0 \mathit{True} \\
  \widehat{\OK}^*(\mktab,\mperm,\epsilon^\mpermmap) &\stepto_1 \passp(\mperm,\mpermmap) \\
  \widehat{\OK}^*(\mktab,\mperm,\kconsm{\mkframe}{\mpermmap}{\mctx}) &\stepto_2 b \text{ where }\\ &\phantom{\stepto_2} b \in \setbuild{\passp(\mperm,\mpermmap) \textbf{ and } b}{
          \makont \in \mktab(\mctx),
          \\&\phantom{\stepto_2 b \in \{}b \in \widehat{\OK}(\mktab,\mperm\setminus\mpermmap^{-1}(\Grant),\makont))}
\end{align*}
This definition works fine with the reentrant $\terminal$ function with a dynamically bound \emph{seen} set, but otherwise needs to be rewritten to accumulate the Boolean result as a parameter of $\widehat{OK}^*$.
%
We use the accumulator version in the proof for simplicity.

\begin{lemma}[Correctness of $\widehat{\OK}$]
  For all $\mktab,\mperm,\mkont,\makont$,
  \begin{itemize}
  \item{\textbf{Soundness:} if $\mkont \in \unroll{\mktab}{\makont}$ then $\OK(\mperm,\mkont) \in \widehat{\OK}(\mktab,\mperm,\makont)$.}
  \item{\textbf{Completeness:} if $b \in \widehat{\OK}(\mktab,\mperm,\makont)$ then there is a $\mkont \in \unroll{\mktab}{\makont}$ such that $b = \OK(\mperm,\mkont)$.}
  \end{itemize}
\end{lemma}
With this lemma in hand, the correctness proof is almost identical to the core proof of correctness.
\begin{theorem}[Correctness]
  The abstract semantics is sound and complete in the same sense as \autoref{thm:pushdown-correct}.
\end{theorem}
%  LocalWords:  pushdown AAM metafunctions metafunction indirected GC
%  LocalWords:  dereferencing idempotence unforgeable
