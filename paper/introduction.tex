Good static analyses use a combination of abstraction techniques, economical data structures, and a lot of engineering~\citep{dvanhorn:CousotEtAl-TASE07,ianjohnson:DBLP:journals/ipl/YiCKK07}.
%
The cited exemplary works stand out from a vast amount of work attacking the problem of statically analyzing languages like C.
%
Dynamic languages do not yet have such gems.
%
The problem space is different, bigger, and full of new challenges.
%
The traditional technique of pushing abstract values around a graph to get an analysis will not work.
%
The first problem we must solve is, ``what graph?'' as control-flow is now part of the problem domain.
%
Second, features like stack inspection and first-class continuations are not easily shoe-horned into a CFG representation of a program's behavior.
%
We need a new approach. % before heavily investing in engineering effort.

%
Luckily, there is an alternative to the CFG approach to analysis construction that is based instead on abstract machines, which are one step away from interpreters; they are interderivable in several instances~\citep{dvanhorn:Danvy:DSc}.
%
This alternative, called abstracting abstract machines
(AAM)~\citep{dvanhorn:VanHorn2012Systematic}, is a simple
idea that is generally applicable to even the most dynamic of
languages, \eg{},
JavaScript~\citep{ianjohnson:DBLP:journals/corr/KashyapDKWGSWH14}.
%
A downside is that all effective instantiations of AAM are finite state approximations.
%
Finite state techniques cannot precisely predict where a method or function call will return.
%
Dynamic languages have more sources for imprecision than non-dynamic languages (\eg{}, reflection, computed fields, runtime linking, {\tt eval}) that all need proper treatment in the abstract.
%
If we can't have precision in the presence of statically unknowable behavior, we should at least be able to \emph{contain} it in the states it actually affects.
%
Imprecise control flow due to finite state abstractions is an unacceptable containment mechanism.
%
It opens the flood gate to imprecision flowing everywhere through analyses' predictions.
%
It is also a solvable problem.
%

%
We extend the AAM technique to computably handle infinite state spaces by adapting pushdown abstraction methods to abstract machines.
%
The unbounded stack of pushdown systems is the mechanism to precisely match calls and returns.
%
We demonstrate the essence of our pushdown analysis construction by
first applying the AAM technique to a call-by-value functional
language (\S\ref{sec:aam}) and then revising the derivation to
incorporate an exact representation of the control stack
(\S\ref{sec:pushdown}).  We then show how the approach scales to
stack-reflecting language features such as garbage collection and
stack inspection (\S\ref{sec:inspection}), and stack-reifying features
in the form of first-class delimited control operators (\S\ref{sec:delim}).
%
These case studies show that the approach is robust in the presence of
features that need to inspect or alter the run-time stack, which
previously have required significant technical
innovations~\cite{dvanhorn:Vardoulakis2011Pushdown,ianjohnson:DBLP:journals/jfp/JohnsonSEMH14}.

Our approach appeals to operational intuitions rather than automata theory to justify the approach.
%
The intention is that the only prerequisite to designing a pushdown
analysis for a dynamic language is some experience implementing
interpreters; expertise in automata theory or abstract domains is
unneeded.
%
By simplifying the technical approach and broadening both the audience
and applicability of pushdown analyses, we hope to enable progress
towards technical feats like
Astr\'ee~\citep{dvanhorn:CousotEtAl-TASE07}
for expressive, dynamic languages.
%
% The large problem domain of analyzing dynamic languages needs an army to tackle it, and we believe that AAM can feed that army.
