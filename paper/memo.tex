All the semantics of previous sections have a performance weakness that many analyses share: unnecessary propagation.
%
Consider two portions of a program that do not affect one another's behavior.
%
Both can change the store, and the semantics will be unaware that the changes will not interfere with the other's execution.
%
The more possible stores there are in execution, the more possible contexts in which a function will be evaluated.
%
Multiple independent portions of a program may be reused with the same arguments and store contents they depend on, but changes to irrelevant parts of the store lead to redundant computation.
%
The idea of skipping from a change past several otherwise unchanged states to uses of the change is called ``sparseness'' in the literature~\citep{ianjohnson:Reif1977Symbolic,ianjohnson:Wegman1991Constant,ianjohnson:DBLP:conf/pldi/OhHLLY12}.
%

%
Memoization is a specialized instance of sparseness; the base stack may change, but the evaluation of the function does not, so given an already computed result we can jump straight to the answer.
%
We use the vocabulary of ``relevance'' and ``irrelevance'' so that future work can adopt the ideas of sparseness to reuse contexts in more ways.
%

Recall the \hyperref[lem:irrelevance]{context irrelevance lemma}: if we have seen the results of a computation before from a different context, we can reuse them.
%
The semantic counterpart to this idea is a memo table that we extend when popping and appeal to when about to push.
%
This simple idea works well with a deterministic semantics, but the non-determinism of abstraction requires care.
%
In particular, memo table entries can end up mapping to multiple results, but not all results will be found at the same time.
%
Note the memo table space:
\begin{align*}
  \mmemo \in \Memo &= \Context \finto \wp(\Relevant) \\
  \Relevant &::= \tpl{\mexpr,\menv,\mstore}
\end{align*}
%
There are a few ways to deal with multiple results:
\begin{enumerate}
\item{rerun the analysis with the last memo table until the table doesn't change (expensive),}
\item{short-circuit to the answer but also continue evaluating anyway (negates most benefit of short-circuiting), or}
\item{use a frontier-based semantics like in \autoref{sec:eng-frontier} with global $\mktab$ and $\mmemo$, taking care to
    \begin{enumerate}
    \item{at memo-use time, still extend $\mktab$ so later memo table extensions will ``flow'' to previous memo table uses, and}
    \item{when $\mktab$ and $\mmemo$ are extended at the same context at the same time, also create states that act like the $\mmemo$ extension point also returned to the new continuations stored in $\mktab$.}
    \end{enumerate}}
\end{enumerate}

We will only discuss the final approach.
%
The same result can be achieved with a one-state-at-a-time frontier semantics, but we believe this is cleaner and more parallelizable.
%
Its second sub-point we will call the ``push/pop rendezvous.''
%
The rendezvous is necessary because there may be no later push or pop steps that would regularly appeal to either (then extended) table at the same context.
%
The frontier-based semantics then makes sure these pushes and pops find each other to continue on evaluating.
%
In pushdown and nested word automata literature, the push to pop short-circuiting step is called a ``summary edge'' or with respect to the entire technique, ``summarization.''
%
We find the memoization analogy appeals to programmers' and semanticists' operational intuitions.
%

%
A second concern for using memo tables is soundness.
%
Without the completeness property of the semantics, memoized results in, \eg{}, an inexactly GC'd machine, can have dangling addresses since the possible stacks may have grown to include addresses that were previously garbage.
%
These addresses would not be garbage at first, since they must be mapped in the store for the contexts to coincide, but during the function evaluation the addresses can become garbage.
%
If they are supposed to then be live, and are used (presumably they are reallocated post-collection), the analysis will miss paths it must explore for soundness.
%

%
Context-irrelevance is a property of the semantics \emph{without} continuation stores, so there is an additional invariant to that of \autoref{sec:pushdown} for the semantics with $\mktab$ and $\mmemo$: $\mmemo$ respects context irrelevance.
\begin{mathpar}
  \inferrule{\dom(\mmemo) \subseteq \dom(\mktab) \\
             \forall \mctx\equiv\tpl{\mexpr_c,\menv_c,\mstore_c} \in \dom(\mmemo),
                     \tpl{\mexpr_r,\menv_r,\mstore_r} \in \mmemo(\mctx), \\
                     \makont\in\mktab(\mctx),
                     \mkont\in\unroll{\mktab}{\makont}. \\
              \exists\mtrace\equiv\tpl{\mexpr_c,\menv_c,\mstore_c,\mkont} \stepto_{\CESKt}^* \tpl{\mexpr_r,\menv_r,\mstore_r,\mkont}. \hastail(\mtrace,\mkont)}
            {\inv_M(\mktab,\mmemo)}
\end{mathpar}
Inexact GC does \emph{not} respect context irrelevance for the same reasons it is not complete: some states are spurious, meaning some memo table entries will be spurious, and the expected path in the invariant will not exist.
%
The reason we use unrolled continuations instead of simply $\epsilon$ for this (balanced) path is precisely for stack inspection reasons.

 \begin{figure}
   \begin{center}
     $\mastate,\mktab,\mmemo \stepto
     \mastate',\mktab',\mmemo'$
     \begin{tabular}{r|l}
       \hline\vspace{-3mm}\\
       $\tpl{\sapp{\mexpri0}{\mexpri1},\menv,\mstore,\makont},\mktab,\mmemo$
       &
       $\tpl{\mexpri0,\menv,\mstore,\kcons{\appl{\mexpri1,\menv}}{\mctx}},\mktab,\mmemo$ \\
       & \quad if $\mctx \notin\dom(\mmemo)$, or \\
       &
       $\tpl{\mexpr',\menv',\mstore',\makont},\mktab',\mmemo$ \\
       & \quad if $\tpl{\mexpr',\menv',\mstore'} \in \mmemo(\mctx)$ \\
       where & $\mctx = \tpl{\sapp{\mexpri0}{\mexpri1},\menv,\mstore}$ \\
       & $\mktab' = \joinm{\mktab}{\mctx}{\makont}$
       \\
       $\tpl{\mval,\mstore,\kcons{\appr{\slam{\mvar}{\mexpr},\menv}}{\mctx}},\mktab,\mmemo$
       &
       $\tpl{\mexpr,\menv',\mstore',\makont},\mktab,\mmemo'$ if $\makont \in \mktab(\mctx)$ \\
       where & $\menv' = \extm{\menv}{\mvar}{\maddr}$ \\
       & $\mstore' = \joinm{\mstore}{\maddr}{\mval}$ \\
       & $\mmemo' = \joinm{\mmemo}{\mctx}{\tpl{\mexpr,\menv',\mstore'}}$
     \end{tabular}
   \end{center}
   \caption{Important memoization rules}
   \label{fig:memo}
 \end{figure}

The rules in \autoref{fig:memo} are the importantly changed rules from \autoref{sec:pushdown} that short-circuit to memoized results.
%
The technique looks more like memoization with a $\CESIKKstart$ machine, since the memoization points are truly at function call and return boundaries.
%
The $\pop$ function would need to also update $\mmemo$ if it dereferences through a context, but otherwise the semantics are updated \emph{mutatis mutandis}.

\begin{equation*}
  {\mathcal F}_{\mexpr}(S,R,F,\mktab,\mmemo) = (S \cup F, R \cup R', F'\setminus S, \mktab', \mmemo')
\end{equation*}
where

\begin{tabular}{rlrlrl}
  $I$ &
  \multicolumn{5}{l}{
    \hspace{-3mm}$=\bigcup\limits_{\mstate \in
      F}{\setbuild{(\tpl{\mstate,\mstate'}, \mktab',\mmemo')}{\mstate,\mktab,\mmemo
        \stepto \mstate',\mktab',\mmemo'}}$}
\\
   $R'$ &\hspace{-3mm}$= \pi_0 I$ & $\mktab'$ & \hspace{-3mm}$= \bigsqcup\pi_1 I$ & $\mmemo'$ & \hspace{-3mm}$= \bigsqcup\pi_2 I$ \\
   $\Delta\mktab$ &\hspace{-3mm}$= \mktab'\setminus\mktab$ & $\Delta\mmemo$ & \hspace{-3mm}$= \mmemo'\setminus\mmemo$ & & \\
   $F'$ &
   \multicolumn{5}{l}{
     \hspace{-3mm}$= \pi_1 R' \cup \{{\tpl{\mexpr,\menv,\mstore,\makont}} :
     {\mctx \in \dom(\Delta\mktab)\cap\dom(\Delta\mmemo).}$}
   \\ &\multicolumn{5}{l}{\hspace{-3mm}$\phantom{= \pi_1 R' \cup \{} \makont \in \Delta\mktab(\mctx),
       \tpl{\mexpr,\menv,\mstore} \in \Delta\mmemo(\mctx)\}$}
 \end{tabular}

\begin{theorem}[Correctness]
Same property is the same as in \autoref{thm:global-pushdown}, where $\reify$ ignores the $\mmemo$ component.
\end{theorem}
The proof appeals to the invariant on $\mmemo$ whose proof involves an additional argument for the short-circuiting step that reconstructs the path from a memoized result using both context irrelevance and the table invariants.
%  LocalWords:  parallelizable pushdown automata summarization
%  LocalWords:  memoization memoized dereferences
