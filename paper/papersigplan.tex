\documentclass[preprint]{sigplanconf}

\pagestyle{plain}
\usepackage{balance,pfsteps}
\usepackage{amsmath,stmaryrd,natbib,listings,graphicx,amsthm,amssymb,mathpartir,mnsymbol}
\usepackage[usenames,dvipsnames]{color}

\newcommand{\Scribtexttt}[1]{{\texttt{#1}}}
\newcommand{\SColorize}[2]{\color{#1}{#2}}
\newcommand{\inColor}[2]{{\Scribtexttt{\SColorize{#1}{#2}}}}
\definecolor{PaleBlue}{rgb}{0.90,0.90,1.0}
\newcommand{\rackett}[1]{\inColor{black}{#1}}
\newcommand{\todo}[1]{\textbf{TODO:} #1}
\newcommand{\iftr}[1]{#1}

\newcommand{\eg}{\textit{e.g.}}
\newcommand{\ie}{\textit{i.e.}}

\newcommand{\nat}{{\mathbb N}}
% Metavariables from defined spaces
\newcommand{\mexpr}{e}
\newcommand{\mexpri}[1]{e_{#1}}
\newcommand{\mvar}{x}
\newcommand{\mvaralt}{y}
\newcommand{\mval}{v}
\newcommand{\mvalpre}{v_\mathit{pre}}
\newcommand{\mvalpost}{v_\mathit{post}}
\newcommand{\mtag}{v_{\mathit{t}}}
\newcommand{\mhandler}{v_\mathit{h}}
\newcommand{\maddr}{a}
\newcommand{\maddralt}{b}
\newcommand{\menv}{\rho}
\newcommand{\mstore}{\sigma}
\newcommand{\mlkont}{\iota}
\newcommand{\mkont}{\kappa}
\newcommand{\mskont}{\hat{\kappa}}
\newcommand{\mmkontacc}{C_\mathit{acc}}
\newcommand{\mmkont}{C}
\newcommand{\mamkont}{\hat{C}}
\newcommand{\mstate}{\varsigma}
\newcommand{\mastate}{\hat\varsigma}

\newcommand{\mtime}{t}
\newcommand{\mtimealt}{u}

\newcommand{\mmarks}{m}
\newcommand{\mmarkset}{\chi}
\newcommand{\mprim}{\mathit{prim}}

\newcommand{\mktab}{\Xi}
\newcommand{\mmktab}{\chi}
\newcommand{\mmemo}{M}
\newcommand{\mpoint}{p}
\newcommand{\mctx}{s}
\newcommand{\msctx}{\hat{\mctx}}
\newcommand{\mframe}{\xi}
\newcommand{\mkframe}{\phi}
\newcommand{\mlab}{\ell}
\newcommand{\mtrace}{\pi}

\newcommand{\callcc}{\rackett{call/cc}}
\newcommand{\callcomp}{\rackett{call/comp}}
\newcommand{\abort}{\rackett{abort}}
\newcommand{\dynamicwind}{\rackett{dynamic-wind}}
\newcommand{\Halt}{\mathbf{Halt}}

% Metavariables for Spaces
\newcommand{\CEK}{\mathit{CEK}}
\newcommand{\CESKstart}{\mathit{CESK^*_t}}
\newcommand{\CESKKstart}{\mathit{CESK^*_t\mktab}}
\newcommand{\CESIKKstart}{\mathit{CESIK^*_t\mktab}}

\newcommand{\Var}{\mathit{Var}}

\newcommand{\Addr}{\mathit{Addr}}
\newcommand{\Time}{\mathit{Time}}

\newcommand{\Expr}{\mathit{Expr}}
\newcommand{\Store}{\mathit{Store}}
\newcommand{\KStore}{\mathit{KStore}}
\newcommand{\LKont}{\mathit{LKont}}
\newcommand{\Kont}{\mathit{Kont}}
\newcommand{\VKont}{\widetilde{\mathit{Kont}}}
\newcommand{\SKont}{\widehat{\mathit{Kont}}}
\newcommand{\MKont}{\mathit{MKont}}
\newcommand{\Env}{\mathit{Env}}
\newcommand{\State}{\mathit{State}}
\newcommand{\System}{\mathit{System}}
\newcommand{\Value}{\mathit{Value}}
\newcommand{\Prims}{\mathit{Prims}}

\newcommand{\maenv}{\hat{\menv}}
\newcommand{\makont}{\hat{\mkont}}
\newcommand{\mvkont}{\tilde{\mkont}}

\newcommand{\Context}{\mathit{Context}}
\newcommand{\SContext}{\mathit{VContext}}

\newcommand{\Point}{\mathit{Point}}
\newcommand{\Frame}{\mathit{Frame}}
\newcommand{\Marks}{\mathit{Marks}}
\newcommand{\Markset}{\mathit{Markset}}

\newcommand{\KTab}{\mathit{Stack}}
\newcommand{\MKTab}{\mathit{KClosure}}
\newcommand{\Memo}{\mathit{Memo}}

% Other metavariables (metafunction names, etc)
\newcommand{\inject}{\mathit{inject}}
\newcommand{\pop}{\mathit{pop}}
\newcommand{\popaux}{\mathit{pop}^*}

\newcommand{\alloc}{\mathit{alloc}}
\newcommand{\tick}{\mathit{tick}}

\newcommand{\bind}{{\mathcal B}}
\newcommand{\lookup}{{\mathcal L}}
\newcommand{\wn}{\mathit{wn}}
\newcommand{\dom}{\text{dom}}
\newcommand{\onep}{\mathit{One?}}
\newcommand{\reverse}{\mathit{reverse}}
\newcommand{\revapp}{\mathit{rev\text{-}append}}
\newcommand{\finalize}{\mathit{finalize}}
\newcommand{\captureupto}{\mathit{capture\text{-}upto}}
\newcommand{\aborttargets}{\mathit{abort\text{-}targets}}

\newcommand{\startstate}{\mathit{base}}
\newcommand{\tails}[2]{\mathit{tails}_{#1}(#2)}
\newcommand{\tailscont}[3]{\mathit{tails}_{#1}^{#2}(#3)}
\newcommand{\reify}{\mathit{reify}}
\newcommand{\reflect}{\mathit{reflect}}
\newcommand{\reachrestrict}{\mathit{unfold}}
\newcommand{\stepextend}{\mathit{unfold}_1}

\newcommand{\domark}{\mathit{mark}}
\newcommand{\domarkset}{\mathit{markset}}
\newcommand{\returns}{{\mathbb R}}
\newcommand{\memoize}{{\mathbb M}}
\newcommand{\approximate}{{\mathbb A}}
\newcommand{\touches}{{\mathcal T}}
\newcommand{\touchesf}{{\mathcal T}_{\mkframe}}
\newcommand{\touchesk}[1]{{\mathcal T}_{#1}}
\newcommand{\reaches}{{\mathcal R}}
\newcommand{\konts}{\mathit{konts}}

\newcommand{\inv}{\mathit{inv}}
\newcommand{\reachable}{\mathit{reach}}
\newcommand{\memo}{\mathit{memo}}
\newcommand{\unroll}[2]{\mathit{unroll}_{#1}(#2)}
\newcommand{\unrollK}[3]{\mathit{unrollK}^{#1}_{#2}(#3)}
\newcommand{\unrollC}[2]{\mathit{unrollC}_{#1}(#2)}
\newcommand{\hastail}{\mathit{ht}}
\newcommand{\replacetail}{\mathit{rt}}

\newcommand{\kontlive}{K{\mathcal L}{\mathcal L}}
\newcommand{\kontliveaux}{K{\mathcal L}{\mathcal L}^*}
\newcommand{\terminal}{\mathit{terminal}}
\newcommand{\terminalaux}{\mathit{terminal}^*}
\newcommand{\trystep}{\mathit{trystep}}
\newcommand{\post}{\mathit{post}}

% Constructors
\newcommand{\svar}[2][{}]{#2^{#1}}
\newcommand{\sapp}[3][{}]{(#2\ #3)#1}
\newcommand{\slam}[3][{}]{\lambda#1 #2. #3}
\newcommand{\sprim}[1]{#1}
\newcommand{\sreset}[1]{(\texttt{reset}\ #1)}
\newcommand{\sshift}[2]{(\texttt{shift}\ #1. #2)}

\newcommand{\swcm}[3]{(\texttt{wcm}\ #1\ #2\ #3)}
\newcommand{\sccms}{(\texttt{ccm})}
\newcommand{\scmsf}{\texttt{cms-first}}
\newcommand{\scmsl}[2]{(\texttt{cms->list}\ #1\ #2)}

\newcommand{\cons}[2]{(\texttt{cons}\ #1\ #2)}
\newcommand{\smarkset}[1]{(\texttt{mark-set}\ #1)}

\newcommand{\vcont}[1]{\mathbf{cont}(#1)}
\newcommand{\vcomp}[1]{\mathbf{comp}(#1)}
\newcommand{\vclo}[2]{(#1, #2)}
\newcommand{\kar}[2][{}]{\mathbf{ar}#1(#2)}
\newcommand{\kpush}[1]{\mathbf{push}(#1)}
\newcommand{\klet}[1]{\mathbf{lt}(#1)}
\newcommand{\kfn}[2][{}]{\mathbf{fn}#1(#2)}
\newcommand{\kwcm}[1]{\mathbf{km}(#1)}
\newcommand{\kwcmk}[1]{\mathbf{kw1}(#1)}
\newcommand{\kwcmv}[1]{\mathbf{kw2}(#1)}
\newcommand{\kcmsls}[1]{\mathbf{kset}(#1)}
\newcommand{\kcmslk}[1]{\mathbf{kkey}(#1)}

\newcommand{\appl}[1]{\mathbf{appL}(#1)}
\newcommand{\appr}[1]{\mathbf{appR}(#1)}


\newcommand{\rt}{\mathbf{rt}}
\newcommand{\krt}[1]{\rt(#1)}
\newcommand{\kmt}{\mathbf{mt}}

\newcommand{\kprompt}[1]{\#(#1)}
\newcommand{\dw}{\mathbf{dw}}
\newcommand{\kdw}[1]{\dw(#1)}
\newcommand{\abrt}{\mathbf{abrt}}
\newcommand{\kabrt}[1]{\abrt(#1)}
\newcommand{\kev}[1]{\mathbf{ev}(#1)}

\newcommand{\mextag}[4]{(\#^{#2}_{#3} #1\ #4)}
\newcommand{\kmatch}[4]{(?^{#2}_{#3} #1\ #4)}
\newcommand{\mkapp}[2]{#1\circ #2}

% Notations
\newcommand{\sa}[1]{\widehat{\mathit{#1}}}

\newcommand{\lfp}{\mathbf{lfp}}
\newcommand{\tpl}[1]{\langle #1 \rangle}
\newcommand{\vect}[1]{\langle #1 \rangle}
\newcommand{\runpost}[1]{\langle #1 \rangle_{\mathit{post}}}
\newcommand{\runhandler}[1]{\langle #1 \rangle_{\mathit{handler}}}
\newcommand{\cstate}[1]{\langle #1 \rangle_{\mathit{call}}}
\newcommand{\set}[1]{\{ #1 \}}
\newcommand{\setbuild}[2]{\{ #1\ :\ #2\}}
\newcommand{\setbuildb}[2]{\left\{ #1\ :\ #2\right\}}
\newcommand{\nequiv}{\centernot\equiv}

\newcommand{\many}[1]{\overline{#1}}
\newcommand{\kcons}[2]{#1{\tt :} #2}
\newcommand{\alt}{\mathrel{\mid}}

\newcommand{\snglm}[2]{[#1 \mapsto \set{#2}]}
\newcommand{\extm}[3]{#1[#2 \mapsto #3]}
\newcommand{\joinm}[3]{#1\sqcup[#2 \mapsto #3]}
\newcommand{\joinone}[3]{#1\sqcup[#2 \mapsto \set{#3}]}
\DeclareMathOperator*{\finto}{\rightharpoonup}
\DeclareMathOperator*{\deceq}{\overset{?}{=}}

\newcommand{\append}[2]{#1 {\tt ++} #2}

%% Relations
\newcommand{\stepto}{\longmapsto}
\newcommand{\pdstepto}{\longmapsto_{\mathit{\Xi{}M}}}
\newcommand{\astepto}{\mathrel{\widehat{\longmapsto}}}
\newcommand{\tred}[1]{\mathrel{\underset{\Pi}{#1}}}

\newcommand{\klectx}[3]{#2 \mathrel{\sqsubseteq_{#1}} #3}
\usepackage{centernot}
\newcommand{\SCodePreSkip}{\vskip\abovedisplayskip}
\newcommand{\SCodePostSkip}{\vskip\belowdisplayskip}
\newenvironment{SCodeFlow}{\SCodePreSkip\begin{list}{}{\topsep=0pt\partopsep=0pt%
\listparindent=0pt\itemindent=0pt\labelwidth=0pt\leftmargin=2ex\rightmargin=2ex%
\itemsep=0pt\parsep=0pt}\item}{\end{list}\SCodePostSkip}
\newenvironment{SingleColumn}{\begin{list}{}{\topsep=0pt\partopsep=0pt%
\listparindent=0pt\itemindent=0pt\labelwidth=0pt\leftmargin=0pt\rightmargin=0pt%
\itemsep=0pt\parsep=0pt}\item}{\end{list}}
\newenvironment{RktBlk}{}{}
\definecolor{IdentifierColor}{rgb}{0.15,0.15,0.50}
\definecolor{ParenColor}{rgb}{0.52,0.24,0.14}
\newcommand{\RktSym}[1]{\inColor{IdentifierColor}{#1}}
\newcommand{\RktPn}[1]{\inColor{ParenColor}{#1}}

\newcommand{\Stttextless}{{\fontencoding{T1}\selectfont<}}
\definecolor{ValueColor}{rgb}{0.13,0.55,0.13}
\newcommand{\RktVal}[1]{\inColor{ValueColor}{#1}}

\newcommand{\ifwcm}[1]{}

\newtheorem{theorem}{Theorem}
\newtheorem{lemma}{Lemma}

\title{Abstracting Abstract Control}
\authorinfo{J. Ian Johnson}{Northeastern University}{ianj@ccs.neu.edu}
\authorinfo{David Van Horn}{University of Maryland}{dvanhorn@cs.umd.edu}

\usepackage{hyperref}
\begin{document}
\maketitle

% Outline:
% Introduction.
% High level
% PDCFA
% - Concrete
% - Abstract
% CFA2
% - Concrete
% - Abstract
% WCM machine
% - Concrete
% - Abstract
% Related Work
% Future Work
% Conclusion

\begin{abstract}
The strengths of dynamic languages are also their weaknesses: these languages' flexibility makes them unpredictable -- statically, that is.
%
Static analyses for dynamic languages are few, far between, and seldom sound.
%
The reasons why range from there being too much unknown before run-time to make useful predictions, to such reusable code that standard data- and control-flow analyses are too imprecise to tease about their different uses, to features that are just too hard to model in traditional analytical frameworks.
%
We propose an analysis construction method that aims to ameliorate the second two of these hurdles.
%

%
Dynamic languages are sometimes referred to as ``interpreted languages'' because some of their features are either too difficult to compile ahead of time, or the features' runtime support resembles the interpreter pattern.
%
The Abstracting Abstract Machines (AAM) method of analysis construction needs little more than a language's interpreter to make a sound approximation of its behavior.
%
AAM thus stands as the simplest road to making analyses for dynamic languages.
%
We improve its precision to better serve dynamic languages by targeting a stronger computational model: pushdown systems.
%
We then show that the techniques scale to precise modeling of delimited, composable first-class control operators, {\tt shift} and {\tt reset}.
%
All of this we do without touching automata theory.
\end{abstract}

\section{Introduction}
\input{introduction-standin}
\section{The AAM methodology}
Abstracting abstract machines is founded on three ideas:
\begin{enumerate}
\item{Recursive data structures are easily representable via indirection at recursive points. For example, the tail of a list goes from being a list to an address of a list in a heap.}
\item{A finite pool of addresses then means a finite state space}
\item{Reused addresses mean the heap must use weak updates: $[a \mapsto v]\sqcup[a \mapsto v'] = [a \mapsto \set{v,v'}]$ and non-deterministic lookups.}
\end{enumerate}

Concretely, let us consider an abstract machine for the call-by-value untyped lambda calculus: the CEK machine.
%
The semantic spaces in \autoref{fig:cek-spaces} have three points of recursion: subexpressions, closures and pushed continuation frames.
%
The semantics for the CEK machine in \autoref{fig:cek-semantics} shows the delayed substitution semantics that environments enable.
%
Function application delays the substitution of an argument by extending the enviroment.
%
Variables dereference the environment to fulfill the delayed substitution semantics that the CEK machine implements.
%
Function application performs an administrative step to search for the next redex.
\begin{figure}\centering  
  \begin{align*}
    \mstate \in \CEK &::= \tpl{\mexpr, \menv, \mkont} \\
    \mexpr \in \Expr &::= \svar{\mvar} \alt \sapp{\mexpr}{\mexpr} \alt \mval \\
    \mval \in \Value &::= \slam{\mvar}{\mexpr} \\
    \menv \in \Env &= \Var \finto \Value \times \Env \\
    \mkframe \in \Frame &::= \appl{\mexpr,\menv} \alt \appr{\mval,\menv} \\
    \mkont \in \Kont &= \Frame^* \\
    \mvar \in \Var & \text{ an infinite space of variables}
  \end{align*}
  \caption{CEK semantic spaces}
\label{fig:cek-spaces}
\end{figure}

\begin{figure}
  \centering
  $\mstate \stepto \mstate'$ \\
  \begin{tabular}{r|l}
    \hline\vspace{-3mm}\\
    $\tpl{\svar\mvar, \menv, \mkont}$
    &
    $\tpl{\mval, \menv', \mkont}$ if $(\mval,\menv') = \menv(\mvar)$
    \\
% Application
    $\tpl{\sapp{\mexpri0}{\mexpri1},\menv,\mkont}$
    &
    $\tpl{\mexpri0,\menv,\kcons{\appl{\mexpri1,\menv}}{\mkont}}$
    \\
% Arg eval
    $\tpl{\mval,\menv, \kcons{\appl{\mexpr,\menv'}}{\mkont}}$
    &
    $\tpl{\mexpr,\menv',\kcons{\appr{\mval,\menv}}{\mkont}}$
    \\
% Function call
    $\tpl{\mval,\menv,\kcons{\appr{\slam{\mvar}{\mexpr},\menv'}}{\mkont}}$
    &
    $\tpl{\mexpr,\menv'[\mvar\mapsto(\mval,\menv)],\mkont}$
  \end{tabular}
  \caption{CEK semantics}
  \label{fig:cek-semantics}
\end{figure}

The recursion in expressions is beneign because they are only destructed. The $\Expr$ space is finite for each program, the size of which is the number of subexpressions.
%
The run-away recursion is in $\Env$ and $\Kont$.
%
AAM suggests that the codomain of $\Env$\footnote{The strict criteria suggests $\Var \finto \Value \times \Addr$, but $\Var \finto \Addr$ is a sound approximation and relates more strongly to other analyses' approximations.} and the tail of the $\Frame$ list in $\Kont$ should instead be some $\Addr$ space.
%
Extensions to maps in $\Env$, and conses of frames in $\Kont$ then instead allocate an address, update the heap with the recursive value, and use the address in place of the recursive value.
%
To help an $\alloc : \State \to \Addr$ function choose its addresses, the state space can be extended with an arbitrary finite space that can be updated each step.
%
The original AAM paper calls this space and update function $\Time$ and $\tick$ respectively.
%
Later work on widening~\citep{ianjohnson:DBLP:conf/vmcai/HardekopfWCK14} suggests less arbitrary constructions, and to think of $\Time$ as a space of abstract traces, though the ``abstraction'' need not be sound~\citep{dvanhorn:Might2009Posteriori}.

The new semantic spaces in \autoref{fig:ceskstart-spaces} form the $\CESKstart$ machine.
%
We represent the machine differently than the original AAM paper to separate induced components (\eg{} the store and $\Time$) from transformed components (\eg{} the environment) for a uniform presentation.
%
The semantics of this machine follow the weak update and non-deterministic lookup principles of AAM in \autoref{fig:ceskstart-semantics}.

\begin{figure}
  \centering
  \begin{align*}
    \mstate \in \sa{CEK} &::= \tpl{\mexpr, \maenv, \makont} \\
    \sa{State} &::= \sa{CEK} \times \Store \times \Time \\
    \mstore \in \Store &= \Addr \to \sa{Storeable} \\
    \menv \in \sa{Env} &= \Var \finto \Addr \\
    \mkont \in \sa{Kont} &::= \epsilon \alt \kcons{\mkframe}{\maddr} \\
    \sa{Storeable} &= \wp(\sa{Kont} + (\Value \times \sa{Env})) \\
    \maddr,\maddralt \in \Addr & \quad \mtime,\mtimealt \in \Time
  \end{align*}
  \caption{$\CESKstart$ semantic spaces}
  \label{fig:ceskstart-spaces}
\end{figure}

\begin{figure}
  \centering
  $\mstate,\mstore,\mtime \stepto \mstate',\mstore',\tick(\mstate,\mstore,\mtime)$ \quad $\maddr = \alloc(\mstate,\mstore,\mtime)$ \\
  \begin{tabular}{r|l}
    \hline\vspace{-3mm}\\
    $\tpl{\svar\mvar, \maenv, \makont}$
    &
    $\tpl{\mval, \maenv',\makont}$ if $(\mval,\menv') \in \mstore(\maenv(\mvar))$
    \\
% Application
    $\tpl{\sapp{\mexpri0}{\mexpri1},\maenv,\makont}$
    &
    $\tpl{\mexpri0,\maenv,\kcons{\appl{\mexpri1,\maenv}}{\maddr}}$ \\
    where & $\mstore' = \joinm{\mstore}{\maddr}{\makont}$
    \\
% Arg eval
    $\tpl{\mval,\maenv, \kcons{\appl{\mexpr,\maenv'}}{\maddralt}}$
    &
    $\tpl{\mexpr,\maenv',\kcons{\appr{\mval,\maenv}}{\maddralt}}$
    \\
% Function call
    $\tpl{\mval,\maenv,\kcons{\appr{\slam{\mvar}{\mexpr},\maenv'}}{\maddralt}}$
    &
    $\tpl{\mexpr,\maenv'',\makont}$ if $\makont \in \mstore(\maddralt)$ \\
    where & $\maenv'' = \maenv'[\mvar\mapsto\maddr]$ \\
          & $\mstore' = \joinm{\mstore}{\maddr}{(\mval,\maenv)}$
  \end{tabular} \\
  Where $\mstore' = \mstore$ unless otherwise stated.
  \caption{$\CESKstart$ semantics}
  \label{fig:ceskstart-semantics}
\end{figure}

If we run the $\CESKstart$ semantics to explore all possible states, we get a sound approximation of all paths that the $\CEK$ machine will explore.
%
The paper will give a more focused view of the $\Kont$ component.
%
Suppose a semantics that interacts with $\Kont$ via a stack discipline (only push and pop single frames at a time, \eg{} the above machine).
%
If we were not to heap-allocate the tail of a continuation, the resulting machine has all finite components except an unbounded stack of finitely many stack frames.
%
This is exactly a pushdown system.
%
We can exactly represent the stack and still have a sound and terminating analysis.
\section{A refinement for exact stacks}\label{sec:pushdown}
We can exactly represent the stack in the $\CESKstart$ machine with a modified allocation scheme for stacks.
%
The key idea is that if the address is ``precise enough,'' then every path that leads to the allocation will proceed exactly the same way until the address is dereferenced.
%

%
What constitutes ``precise enough?'' 
%
For the $\CESKstart$ machine, every function evaluates the same way, regardless of the stack.
%
We should then represent the stack addresses as the state that allocates it, without the rest of the stack.
%
We store stack addresses in a different table because execution is disregards the rest of the stack.
%
We will call this table $\mktab$, because it looks like a stack.
%

%
By not storing the continuations in the value store, we separate ``relevant'' components from ``irrelevant'' components.
%
We split the stack store from the value store and use only the value store in stack addresses.
%
Stack addresses generally describe the relevant context that lead to their allocation, so we will refer to them henceforth as \emph{contexts}.
%
The resulting state space is updated here:
  \begin{align*}
    \sa{State} &= \sa{CEK} \times \Store \times \Time \times \KStore \\
    \sa{Storeable} &= \wp(\Value \times \sa{Env}) \\
    \mkont \in \Kont &::= \epsilon \alt \kcons{\mkframe}{\mctx} \\
    \mctx \in \Context &::=  \tpl{\mexpr,\maenv,\mstore,\mtime} \\
    \mktab \in \KStore &= \Context \finto \wp(\Kont) \\
  \end{align*}

The semantics is modified slightly in \autoref{fig:ceskkstart-semantics} to use $\mktab$ instead of $\mstore$ for continuation allocation and lookup.
%
Given finite allocation, contexts are drawn from a finite space, but are still precise enough to describe an unbounded stack.
%
The computed $\stepto$ relation thus represents the full description of a pushdown system of reachable states (and the set of paths).
%
Of course this semantics does not always define a pushdown system since $\alloc$ can have an unbounded codomain.
%
The correctness claim is therefore a correspondence between the same machine but with an unbounded stack, no $\mktab$, and an $\alloc$ function that behaves the same disregarding the different representations (a reasonable assumption).

\begin{figure}
  \centering
  $\mastate,\mstore,\mtime,\mktab \stepto \mastate',\mstore',\tick(\mastate,\mstore,\mtime),\mktab'$ \quad $\maddr = \alloc(\mastate,\mstore,\mtime)$ \\
  \begin{tabular}{r|l}
    \hline\vspace{-3mm}\\
    $\tpl{\svar\mvar, \maenv, \makont}$
    &
    $\tpl{\mval, \maenv',\makont}$ if $(\mval,\menv') \in \mstore(\maenv(\mvar))$
    \\
% Application
    $\tpl{\sapp{\mexpri0}{\mexpri1},\maenv,\makont}$
    &
    $\tpl{\mexpri0,\maenv,\kcons{\appl{\mexpri1,\maenv}}{\mctx}}$ \\
    where & $\mctx = \tpl{\sapp{\mexpri}{\mexpri1},\maenv,\mstore,\mtime}$ \\
          & $\mktab' = \joinm{\mktab}{\mctx}{\makont}$
    \\
% Arg eval
    $\tpl{\mval,\maenv, \kcons{\appl{\mexpr,\maenv'}}{\mctx}}$
    &
    $\tpl{\mexpr,\maenv',\kcons{\appr{\mval,\maenv}}{\mctx}}$
    \\
% Function call
    $\tpl{\mval,\maenv,\kcons{\appr{\slam{\mvar}{\mexpr},\maenv'}}{\mctx}}$
    &
    $\tpl{\mexpr,\maenv'',\makont}$ if $\makont \in \mktab(\mctx)$ \\
    where & $\maenv'' = \maenv'[\mvar\mapsto\maddr]$ \\
          & $\mstore' = \joinm{\mstore}{\maddr}{(\mval,\maenv)}$
  \end{tabular} \\
  Where $\mstore' = \mstore$ and $\mktab' = \mktab$ unless otherwise stated.
  \caption{$\CESKKstart$ semantics}
  \label{fig:ceskkstart-semantics}
\end{figure}

\subsection{Correctness}

The high level argument for correctness exploits properties of both machines.
%
Where the stack is unbounded (call this $\CESKt$), any trace where states share a common tail in their continuations, that tail can be replaced with anything and the trace will still be valid.
%
We call this property more generally, ``context irrelevance.''
%
The second machine maintains an invariant on $\mktab$ that essentially says if $\makont \in \mktab(\mctx)$, then there is a trace in the first machine that starts at the tail of $\makont$ and reaches $\mctx$ with $\makont$ on top.
%
We can use this invariant and context irrelevance to translate steps in the second machine into steps of the first.
%
The other way around, we use a proposition that a full stack is represented by $\mktab$ via unrolling.

The common tail proposition we will call $\hastail$ and the replacement function we will call $\replacetail$; they both have obvious inductive and recursive definitions respectively.
%
The invariant is stated with respect to the entire program, $\mexpr_\mathit{pgm}$:
\begin{mathpar}
  \inferrule{ }{\invmktab(\bot)} \quad
  \inferrule{\invmktab(\mktab) \\
      \forall \makont_c \in K. \startstate(\makont_c) \stepto_\CESKt^* \tpl{\mexpr_c,\maenv_c,\append{\mkont_c}{\epsilon}},\mstore_c,\mtime_c}
            {\invmktab(\mktab[\tpl{\mexpr_c,\maenv_c,\mstore_c,\mtime_c} \mapsto K])} \\

  \inferrule{
    % invariant for current state
    \startstate(\makont) \stepto_\CESKt^* \tpl{\mexpr,\maenv,\append{\makont}{\epsilon}},\mstore,\mtime \\
    % invariant for all entries
    \invmktab(\mktab)}
    {\inv(\tpl{\mexpr,\maenv,\makont},\mstore,\mktab,\mtime)}
  \end{mathpar}
where
\begin{align*}
 \startstate(\epsilon) &= \tpl{\mexpr_\mathit{pgm},\bot,\epsilon},\bot,\mtime_0 \\
                \startstate(\kcons{\mkframe}{\tpl{\mexpr_c,\maenv_c,\mstore_c,\mtime_c}}) &=
                \tpl{\mexpr_c,\maenv_c,\epsilon},\mstore_c,\mtime_c
\end{align*}
We use $\append{\cdot}{\epsilon}$ to treat $\mctx$ like $\epsilon$ and construct a continuation in $\Kont$ rather than $\sa{Kont}$.
%%
\begin{lemma}[Context irrelevance]\label{lem:irrelevance}
  For all traces $\mtrace \in \CESKt^*$ and continuations $\mkont$ such that $\hastail(\mtrace,\mkont)$, for any $\mkont'$, $\replacetail(\mtrace,\mkont,\mkont')$ is also a valid trace.
\end{lemma}
\begin{proof}
  Simple induction on $\mtrace$ and cases on $\stepto_{\CESKt}$.
\end{proof}
\begin{lemma}[$\CESKKstart$ Invariant]\label{lem:invariant}
  For all $\mstate,\mstate' \in \sa{State}$, if $\inv(\mstate)$ and $\mstate \stepto \mstate'$, then $\inv(\mstate')$
\end{lemma}
\begin{proof}
  Routine case analysis.
\end{proof}

The unrolling proposition is the following
\begin{mathpar}
  \inferrule{ }{\epsilon \in \unroll{\mktab}{\epsilon}} \quad
  \inferrule{\makont \in \mktab(\mctx),
             \mkont \in \unroll{\mktab}{\makont}}
            {\kcons{\mkframe}{\mkont} \in \unroll{\mktab}{\kcons{\mkframe}{\mctx}}}
\end{mathpar}
\begin{theorem}[Correctness]
  For all expressions $\mexpr_\mathit{pgm}$,
  \begin{itemize}
  \item{%for all $\mstate\equiv\tpl{\mexpr,\maenv,\mkont},\mstate'\equiv{\mexpr',\maenv',\mkont'} \in \CESKt$,
        if $\mstate,\mstore,\mtime \stepto_{\CESKt} \mstate',\mstore',\mtime'$,
        for all $\mktab,\makont$ if $\inv(\mstate\set{\mkont := \makont},\mstore,\mktab,\mtime)$
        and $\mkont \in \unroll{\mktab}{\makont}$, then
        there is a $\mktab',\makont'$ such that
        $\mstate\set{\mkont := \makont},\mstore,\mktab,\mtime \stepto_{\CESKKstart} \mstate'\set{\mkont := \makont'},\mstore',\mktab',\mtime'$ and $\mkont' \in \unroll{\mktab'}{\makont'}$}
  \item{if $\mastate,\mstore,\mktab,\mtime \stepto_{\CESKKstart} \mastate',\mstore',\mktab',\mtime'$
      and $\inv(\mastate,\mstore,\mktab,\mtime)$,
      for all $\mkont$, if $\mkont \in \unroll{\mktab}{\mastate.\makont}$ then
      there exists an $\mkont'$ such that
      $\mastate\set{\makont := \mkont},\mstore,\mtime \stepto_{\CESKt}
       \mastate'\set{\makont := \mkont'},\mstore',\mtime'$ and
       $\mkont' \in \unroll{\mktab}{\mastate'.\makont}$.}
  \end{itemize}
\end{theorem}

\subsection{Engineered semantics for efficiency}
%%
There are three optimizations that may be employed to accelerate the fixed-point computation.
\begin{enumerate}
\item{Observe that $\mktab$ can be made global with no loss in precision; it will not need to be stored in the frontier or set of seen states.}
\item{Continuations can be ``chunked'' more coarsely at function boundaries instead of at each frame in order to minimize table lookups.}
\item{Since evaluation is the same regardless of the stack, we can memoize results to short-circuit to the answer.}
\end{enumerate}
%
This last optimization will be covered in more detail in \autoref{sec:memo}.

Flow analyses commonly split control-flow graphs at function call boundaries to enable the combination of intra- and inter-procedural analyses.
%
In an abstract machines, this split looks like installing a continuation prompt at function calls.
%
We borrow a representation from literature on delimited continuations to split the continuation into two components: the continuation and meta-continuation.
%
Our delimiters are special since each continuation ``chunk'' until the next prompt will be of bounded length.
%
The bound is roughly the deepest nesting depth of an expression in functions' bodies.
%
Instead of ``continuation'' and ``meta-continuation'' then, we will use terminology from CFA2 and call the top chunk a ``local continuation,'' and the rest the ``continuation.''
%
%

\begin{figure}
  \centering
  \begin{align*}
    \mstate \in \sa{CEIK} &= \tpl{\mexpr,\menv,\mlkont,\mkont} \\
    \mlkont \in \LKont &= \Frame^* \\
    \makont \in \Kont &= \Context + \set{\Halt}
  \end{align*}
  \caption{$\CESIKKstart$ semantic spaces}
  \label{fig:pushdown-spaces}
\end{figure}

The resulting shuffling of the semantics to accommodate this new representation is in \autoref{fig:cesikkstart-semantics}.
%
The extension to $\mktab$ happens in a different rule -- function entry -- so the shape of the context changes to hold the function, argument, store and time.
%
We have a choice of whether to introduce an administrative step to dereference $\mktab$ once $\mlkont$ is empty, or to use a helper metafunction to describe a ``pop'' of both $\mlkont$ and $\mkont$.
%
Suppose we choose the second because the resulting semantics has a 1-to-1 correspondence with the previous semantics.
%
A first try might land us here:
\begin{align*}
  \pop(\kcons{\mkframe}{\mlkont},\makont,\mktab) &= \set{(\mkframe,\mlkont,\makont)} \\
  \pop(\epsilon,\mctx,\mktab) &= \setbuild{(\mkframe,\mlkont,\makont)}{(\kcons{\mkframe}{\mlkont}, \makont) \in \mktab(\mctx)}
\end{align*}
However, tail calls make the dereferenced $\mctx$ lead to $(\epsilon,\mctx')$.
%
Because abstraction makes the store grow monotonically in a finite space, it's possible that $\mctx' = \mctx$ and a naive recursive definition of $\pop$ will diverge chasing these contexts.
%
Now $\pop$ must save all the contexts it dereferences in order to guard against divergence.
%
So $\pop(\mlkont,\makont,\mktab) = \popaux(\mlkont,\makont,\mktab,\emptyset)$ where
\begin{align*}
  \popaux(\epsilon,\epsilon,\mktab,G) &= \emptyset \\
  \popaux(\kcons{\mkframe}{\mlkont},\makont,\mktab,G) &= \set{(\mkframe,\mlkont,\makont)} \\
  \popaux(\epsilon,\mctx,\mktab,G) &= \setbuild{(\mkframe,\mlkont,\makont)}{(\kcons{\mkframe}{\mlkont}, \makont) \in \mktab(\mctx)} \\
  &\cup \bigcup\limits_{\mctx' \in G'}\popaux(\epsilon,\mctx',\mktab,G\cup G') \\
  \text{where } G' &= \setbuild{\mctx'}{(\epsilon,\mctx') \in \mktab(\mctx)} \setminus G
\end{align*}

In practice, one would not expect $G$ to grow very large.
%
Had we chosen the first strategy, the issue of divergence is delegated to the machinery from the fixed-point computation.\footnote{CFA2 employs the first strategy and calls it ``transitive summaries.''}
%
However, when adding the administrative state, the ``seen'' check requires searching a far larger set than we would expect $G$ to be.

\begin{figure}
  \centering
  $\mastate,\mstore,\mtime,\mktab \stepto \mastate',\mstore',\tick(\mastate,\mstore,\mtime),\mktab'$ \quad $\maddr = \alloc(\mstate,\mstore,\mtime)$ \\
  \begin{tabular}{r|l}
    \hline\vspace{-3mm}\\
    $\tpl{\svar\mvar, \maenv, \mlkont, \makont}$
    &
    $\tpl{\mval, \maenv',\mlkont,\makont}$ if $(\mval,\menv') \in \mstore(\maenv(\mvar))$
    \\
% Application
    $\tpl{\sapp{\mexpri0}{\mexpri1},\maenv,\mlkont,\makont}$
    &
    $\tpl{\mexpri0,\maenv,\kcons{\appl{\mexpri1,\maenv}}{\mlkont},\makont}$
    \\
% Arg eval
    $\tpl{\mval,\maenv, \mlkont,\makont}$
    &
    $\tpl{\mexpr,\maenv',\kcons{\appr{\mval,\maenv}}{\mlkont'},\makont'}$ \\
    &
    if $\appl{\mexpr,\maenv'}, \mlkont',\makont' \in \pop(\mlkont,\makont,\mktab)$ \\
% Function call
    $\tpl{\mval,\maenv, \mlkont,\makont}$
    &
    $\tpl{\mexpr,\maenv'[\mvar\mapsto\maddr],\epsilon,\mctx}$ \\
    & if $\appr{\slam{\mvar}{\mexpr},\maenv'}, \mlkont', \makont' \in \pop(\mlkont,\makont,\mktab)$ \\
    where & $\mstore' = \joinm{\mstore}{\maddr}{(\mval,\maenv)}$ \\
    & $\mctx = (\mval,\maenv,\slam{\mvar}{\mexpr},\maenv',\mstore,\mtime)$ \\
    & $\mktab' = \joinm{\mktab}{\mctx}{(\mlkont,\makont)}$
  \end{tabular} \\
  Where $\mstore' = \mstore$ and $\mktab' = \mktab$ unless otherwise stated.
  \caption{$\CESIKKstart$ semantics}
  \label{fig:cesikkstart-semantics}
\end{figure}

\begin{align*}
  {\mathcal F}(S,R,F,\mktab) &= (S \cup F, R \cup R', F'\setminus S, \mktab') \\
  I &= \bigcup\limits_{s=(\mstate,\mstore,\mtime) \in F}{\setbuild{(\tpl{s,s'}, \mktab')}{s,\mktab \stepto s',\mktab'}} \\
  R' &= \pi_0 I \\
  F' &= \pi_1 R' \\
  \mktab' &= \bigsqcup\pi_1 I
\end{align*}
For a program $\mexpr$, we will say $(\emptyset,\emptyset,\set{\tpl{\mexpr,\bot,\epsilon,\Halt}},\bot)$ is the bottom element of ${\mathcal F}$'s domain.
%
The ``analysis'' then is then the pair of the $R$ and $\mktab$ components of $\lfp({\mathcal F})$.

\begin{theorem}[Correctness]
  \todo{Ian}
\end{theorem}
\section{Stack inspection and recursive metafunctions}
Since we just showed how to produce a pushdown system from an abstract machine, some readers may be concerned that we have lost the ability to reason about the stack as a whole.
%
This is not the case.
%
The semantics may still refer to $\mktab$ to make judgments about the possible stacks that can be realized at each state.
%
The key is to interpret the functions making these judgments again with the AAM methodology.

%
Some semantic features allow a language to inspect some arbitrarily deep part of the stack, or compute a property of the whole stack before continuing.
%
Java's access control security features are an example of the first form of inspection, and garbage collection is an example of the second.
%
We will demonstrate both forms are simple first-order metafunctions that the AAM methodology will soundly interpret.
%
Access control can be modeled with continuation marks, so we demonstrate with the CM machine of \citeauthor{dvanhorn:Clements2004Tailrecursive}.

%%
Semantics that inspect the stack do so with metafunction calls that recur down the stack.
%
Recursive metafunctions can be thought of as out-of-band reduction relations that match rules in order until one fires, repeatedly until a return value is produced.
%
Interpreted via AAM, non-deterministic metafunction evaluation leads to a set of possible results.
%
The finite restriction on the state space carries over to metafunction inputs, so we can always detect infinite loops and bail out of that execution path.
%
Specifically, a metafunction call can be seen as an initial state, $s$, in a reduction system for which we can compute all terminal states:
\begin{align*}
  \terminal &: \forall A. (\text{relation } A)^* \times A \to \wp(A) \\
  \terminal(\mathit{Rs},s) &= \terminalaux(\set{s},\set{s},\emptyset) \\[2pt]
  \text{where } \terminalaux(S, \emptyset, T) &= T \\
   \terminalaux(S, F \uplus \set{s}, T) &= \terminalaux(S\cup I, F \cup (I \setminus S), T') \\
   & \text{where } (I',T') = \trystep(\mathit{Rs}) \\
   \trystep(\epsilon) &= (\emptyset,T\cup\set{s}) \\
   \trystep(\stepto:\mathit{Rs}) &= I\deceq\emptyset \to \trystep(\mathit{Rs}), (I, T) \\
    & \text{where } I = \setbuild{s'}{s \stepto s'}
\end{align*}

This definition is a typical worklist algorithm.
%
It builds the set of terminal terms, $T$, by exploring the frontier (or worklist), $F$, and only adding terms to the frontier that have not been seen, as represented by $S$.
%
If a rule does not match a given term, the intermediate set, $I$ will be empty, and $\terminalaux$ will continue trying the following rules until it finds a match.
%
It is possible for metafunctions' rewrite rules to themselves use metafunctions, but the \emph{seen} set for $\terminal$ must be dynamically bound\footnote{This is a reference to dynamic scope as opposed to lexical scope.} -- it cannot restart at $\emptyset$ upon reentry.
%
Without this precaution, the host language will exceed its stack limits when an infinite path is explored, rather than bail out.

\subsection{Pushdown GC}
Garbage collection, for example, needs to crawl the stack for live addresses.
%
We are interested in garbage collection because it can give massive precision boosts to analyses~\citep{dvanhorn:Might:2006:GammaCFA,dvanhorn:Earl2012Introspective}.
%
The following function will produce the set of live addresses in the stack:

\begin{align*}
  \kontlive &: \Frame^* \to \wp(\Addr) \\
  \kontlive(\makont) &= \kontliveaux(\makont,\emptyset) \\[2pt]
  \kontliveaux(\epsilon,L) &= L \\
  \kontliveaux(\kcons{\mkframe}{\makont}, L) &= \kontliveaux(\makont, L\cup\touches(\mkframe)) \\
  \text{where } \touches(\appl{\mexpr,\menv}) &= \touches(\appr{\mexpr,\menv}) = \touches(\mexpr,\menv) \\
                \touches(\mexpr,\menv) &= \setbuild{\menv(\mvar)}{\mvar \in \fv(\mexpr)}
\end{align*}

When interpreted via AAM, the continuation is indirected through $\mktab$ and leads to multiple results, and possibly loops through $\mktab$.
%
Thus this is more properly understood as
\begin{align*}
  \kontlive(\mktab,\makont) &= \terminal(\set{\stepto_i}_{i=0}^1, \kontliveaux(\mktab,\makont,\emptyset)) \\[2pt]
  \kontliveaux(\mktab,\epsilon,L) &\stepto_0 L \\
  \kontliveaux(\mktab,\kcons{\mkframe}{\mctx}, L) &\stepto_1 \kontliveaux(\mktab,\makont, L\cup\touches(\mkframe)) \text{ if } \makont \in \mktab(\mctx)
\end{align*}
%

A garbage collecting semantics can choose to collect the heap with respect to each live set (call this $\Gamma^*$), or, soundly, collect with respect to their union (call this $\hat\Gamma$).\footnote{The garbage collecting version of PDCFA~\citep{ianjohnson:DBLP:journals/jfp/JohnsonSEMH14} evaluates the $\hat\Gamma$ strategy.}
%
On the one hand we could have tighter collections but more possible states, and on the other hand we can leave some precision behind in the hopes that the state space will be smaller.
%
In the general idea of relevance versus irrelevance, the continuation's live addresses are relevant to execution, but are already implicitly represented in contexts because they must be mapped in the store's domain.

%
A state is ``collected'' only if live addresses remain in the domain of $\mstore$.
%
We say a value $\mval \in \mstore(\maddr)$ is live if $\maddr$ is live.
%
If a value is live, any addresses it touches are live; this is captured by the computation in $\reaches$:
%
\begin{align*}
  \reaches(\mathit{root},\mstore) &=
 \setbuild{\maddralt}{\maddr \in \mathit{root}, \maddr \leadsto_\mstore^* \maddralt} \\
&  \infer{\mval \in \mstore(\maddr) \\ \maddralt \in \touches(\mval)}{\maddr \leadsto_\mstore \maddralt}
\end{align*}
So the two collection methods are as follows.
%
Exact GC produces different collected states based on the possible stacks' live addresses:\footnote{It is possible and more efficient to build the stack's live addresses piecemeal as an additional component of each state, precluding the need for $\kontlive$. Each stack in $\mktab$ would also store the live addresses to restore on pop.}
\begin{align*}
  \Gamma^*(\mastate,\mktab) &=
    \setbuild{\mastate\set{\mstore:=\mastate.\mstore|_L}}{L \in \live^*(\mastate,\mktab)} \\
  \live^*(\tpl{\mexpr,\menv,\mstore,\makont},\mktab) &=
    \setbuild{\reaches(\touches(\mexpr,\menv) \cup L, \mstore)}{L \in \kontlive(\mktab,\makont)}
\end{align*}
\begin{equation*}
  \infer{\mastate,\mktab \stepto \mastate',\mktab' \\
         \mastate' \in \Gamma^*(\mstate',\mktab')}
        {\mastate,\mktab \stepto_{\Gamma^*} \mastate',\mktab}
\end{equation*}
And inexact GC produces a single state that collects based on all (known) stacks' live addresses:
\begin{align*}
  \hat\Gamma(\mastate,\mktab) &=
  \mastate\set{\mstore:=\mastate.\mstore|_{\widehat{\live}(\mastate,\mktab)}} \\
  \widehat{\live}(\tpl{\mexpr,\menv,\mstore,\makont},\mktab) &=
    \reaches(\touches(\mexpr,\menv) \cup \bigcup\kontlive(\mktab,\makont), \mstore)
\end{align*}
\begin{equation*}
  \infer{\mastate,\mktab \stepto \mastate',\mktab'}
        {\mastate,\mktab \stepto_{\hat\Gamma} \hat\Gamma(\mstate',\mktab')\mktab'}
\end{equation*}

Without the continuation store, the baseline GC is
\begin{align*}
  \Gamma(\mastate) &= \mastate\set{\mstore:=\mastate.\mstore|_{\live(\mastate)}} \\
  \live(\mexpr,\menv,\mstore,\mkont) &= \reaches(\touches(\mexpr,\menv)\cup \kontlive(\makont), \mstore)
\end{align*}
\begin{equation*}
  \infer{\mastate \stepto \mastate'}
        {\mastate \stepto_{\Gamma} \Gamma(\mstate')}  
\end{equation*}
Suppose at arbitrary times we decide to perform garbage collection rather than continue with garbage.
%
So when $\mastate \stepto \mastate'$, we instead do $\mastate \stepto_\Gamma \mastate'$.
%
The times we perform GC do not matter for soundness, since we are not analyzing GC behavior.
%
However, garbage stands in the way of completeness.
%
Mismatches in the GC application for the different semantics lead to mismatches in resulting state spaces, not just up to garbage in stores, but in spurious paths from dereferencing a reallocated address that was not first collected.
%

%
The state space compaction that continuation stores give us makes ensuring GC times match up for the completeness proposition tedious.
%
Our statement of completeness then will assume both semantics perform garbage collection on every step.
%

\begin{lemma}[Correctness of $\kontlive$]
  For all $\mktab,\mkont,\makont$,
  \begin{itemize}
  \item{\textbf{Soundness:} if $\mkont \in \unroll{\mktab}{\makont}$ then $\kontlive(\mkont) \in \kontlive(\mktab,\mkont)$}
  \item{\textbf{Completeness:} for all $L \in \kontlive(\mktab,\makont)$ there is a $\mkont \in \unroll{\mktab}{\makont}$ such that $L = \kontlive(\mkont)$.}
  \end{itemize}
\end{lemma}
\begin{proof}
  Soundness follows by induction on the unrolling. Completeness follows by \todo{Ian}.
\end{proof}
\begin{theorem}[Correctness of $\Gamma^*\CESKKstart$]
  For all expressions $\mexpr_0$,
  \begin{itemize}
  \item{{\bf Soundness: } %for all $\mstate\equiv\tpl{\mexpr,\menv,\mkont},\mstate'\equiv{\mexpr',\menv',\mkont'} \in \CESKt$,
        if $\mstate \stepto_{\Gamma\CESKt} \mstate'$,
        %for all $\mktab,\makont$ if
        $\inv(\mstate\set{\mkont := \makont},\mktab)$,
        and $\mstate.\mkont \in \unroll{\mktab}{\makont}$, then
        there are $\mktab',\makont',\mstore'$ such that
        $\mstate\set{\mkont := \makont},\mktab \stepto_{\Gamma^*\CESKKstart} \mastate',\mktab'$ where
        $\mastate' = \mstate'\set{\mkont := \makont',\mstore:=\mstore'}$ 
        and $\mstate'.\mkont \in \unroll{\mktab'}{\makont'}$
        and finally there is an $L \in \live^*(\mastate',\mktab')$ such that $\mstore'|_L = \mstate'.\mstore|_{\live(\mstate')}$}
  \item{{\bf Completeness:} if $\mastate\equiv\tpl{\mexpr,\menv,\mstore,\makont},\mktab \stepto_{\Gamma^*\CESKKstart} \mastate',\mktab'$ and there is an $L_\mkont \in \kontlive(\mktab,\makont)$ such that $\mstore|_L = \mstore$ (where $L = \reaches(\touches(\mexpr,\menv) \cup L_\mkont, \mstore)$) and $\inv(\mastate,\mktab)$,
      for all $\mkont \in \unroll{\mktab}{\makont}$ such that $\kontlive(\mkont) = L_\mkont$,
      there is a $\mkont'$ such that
      $\mastate\set{\makont := \mkont} \stepto_{\Gamma\CESKt}
      \mastate'\set{\makont := \mkont'}$ (a GC step) and
      $\mkont' \in \unroll{\mktab}{\mastate'.\makont}$}
  \end{itemize}  
\end{theorem}

\begin{theorem}[Soundness of $\hat\Gamma\CESKKstart$]
  For all expressions $\mexpr_0$,
  if $\mstate \stepto_{\Gamma\CESKt} \mstate'$,
  % for all $\mktab,\makont$ if
  $\inv(\mstate\set{\mkont := \makont},\mktab)$,
  and $\mstate.\mkont \in \unroll{\mktab}{\makont}$, then
  there are $\mktab',\makont',\mstore''$ such that
  $\mstate\set{\mkont := \makont},\mktab \stepto_{\hat\Gamma\CESKKstart} \mastate',\mktab'$ where
  $\mastate' = \mstate'\set{\mkont := \makont',\mstore:=\mstore''}$
  and $\mstate'.\mkont \in \unroll{\mktab'}{\makont'}$
  and finally $\mstate'.\mstore|_{\live(\mstate')} \sqsubseteq \mstore''|_{\widehat{live}(\mastate',\mktab')}$
\end{theorem}
The proofs are straightforward, and use the usual lemmas for GC, such as idempotence of $\Gamma$ and $\touches \subseteq \reaches$.
\subsection{Analyzing security features: the CM machine}
The CM machine provides a model of access control: a dynamic branch of execution is given permission to use some resource if a continuation mark for that permission is set to ``grant.''
%
There are three new forms we add to the lambda calculus to model this feature: {\tt grant}, {\tt frame}, and {\tt test}.
%
The {\tt grant} construct adds a permission to the stack.
%
The concern of unforgeable permissions is orthogonal, so we simplify with a set of permissions that is textually present in the program:
\begin{align*}
  \mperm \in \Permissions & \text{ a set} \\
  \Expr &::= \ldots \alt \sgrant{\mperm}{\mexpr} \alt \sframe{\mperm}{\mexpr} \alt \stest{\mperm}{\mexpr}{\mexpr}
\end{align*}
%
The {\tt frame} construct ensures that only a given set of permissions are allowed, but not necessarily granted.
%
The security is in the semantics of {\tt test}: we can test if all marks in some set $\mperm$ have been granted in the stack without first being denied; this involves crawling the stack:
\begin{align*}
  \OK(\emptyset,\mkont) &= \mathit{True} \\
  \OK(\mperm,\epsilon^\mpermmap) &= \passp(\mperm,\mpermmap) \\
  \OK(\mperm,\kconsm{\mkframe}{\mpermmap}{\mkont}) &= \passp(\mperm,\mpermmap) \textbf{ and } \OK(\mperm \setminus \mpermmap^{-1}(\Grant), \mkont) \\
  \text{where }\passp(\mperm,\mpermmap) &= \mperm \cap \mpermmap^{-1}(\Deny) \deceq \emptyset
\end{align*}
The set subtraction is to say that granted permissions do not need to be checked farther down the stack.
%

%
Continuation marks respect tail calls and have an interface that abstracts over the stack implementation.
%
Each stack frame added to the continuation carries the permission map.
%
The empty continuation also carries a permission map.
%
Crucially, the added constructs do not add frames to the stack; instead, they update the permission map in the top frame, or if empty, the empty continuation's permission map.
\begin{align*}
  \mpermmap \in \PermissionMap &= \Permissions \finto \GD \\
  \mgd \in \GD &::= \Grant \alt \Deny \\
  \mkont \in \Kont &::= \epsilon^\mpermmap \alt \kconsm{\mkframe}{\mpermmap}{\mkont}
\end{align*}
Update for continuation marks:
\begin{align*}
  \extm{\mpermmap}{\mperm}{\mgd} &= \lambda x. x \decin \mperm \to \mgd, \mpermmap(x) \\
  \extm{\mpermmap}{\overline{\mperm}}{\mgd} &= \lambda x. x \decin \mperm \to \mpermmap(x),\mgd \end{align*}

\begin{figure}
  \centering
  \begin{tabular}{r|l}
    \hline\vspace{-3mm}\\
    $\tpl{\sgrant{\mperm}{\mexpr}, \menv, \mstore, \mkont}$
    &
    $\tpl{\mexpr,\menv,\mstore, \extm{\mkont}{\mperm}{\Grant}}$
    \\
    $\tpl{\sframe{\mperm}{\mexpr}, \menv,\mstore,  \mkont}$
    &
    $\tpl{\mexpr,\menv,\mstore, \extm{\mkont}{\overline{\mperm}}{\Deny}}$
    \\
    $\tpl{\stest{\mperm}{\mexpri0}{\mexpri1}, \menv,\mstore,  \mkont}$
    &
    $\tpl{\mexpri0,\menv,\mstore, \mkont}$ if $\mathit{True} = \OK(\mperm,\mkont)$
    \\
    &
    $\tpl{\mexpri1,\menv,\mstore, \mkont}$ if $\mathit{False} = \OK(\mperm,\mkont)$
  \end{tabular}
  \caption{CM machine semantics}
  \label{fig:cm-semantics}
\end{figure}

The abstract version of the semantics has one change on top of the usual continuation store.
%
The {\tt test} rules are now
\begin{align*}
  \tpl{\stest{\mperm}{\mexpri0}{\mexpri1}, \menv, \mstore, \makont},\mktab
  &\stepto
  \tpl{\mexpri0,\menv,\mstore, \makont},\mktab \text{ if } \mathit{True} \in \widehat{\OK}(\mktab,\mperm,\makont)
  \\
  &\stepto
  \tpl{\mexpri1,\menv,\mstore,\makont},\mktab \text{ if } \mathit{False} \in \widehat{\OK}(\mktab,\mperm,\makont)
\end{align*}
where the a new $\widehat{\OK}$ function uses $\terminal$ and rewrite rules:
\begin{align*}
  \widehat{\OK}(\mktab,\mperm,\makont) &= \terminal(\set{\stepto_i}_{i=0}^2,\widehat{\OK}^*(\mktab,\mperm,\makont)) \\[2pt]
  \widehat{\OK}^*(\mktab,\emptyset,\makont) &\stepto_0 \mathit{True} \\
  \widehat{\OK}^*(\mktab,\mperm,\epsilon^\mpermmap) &\stepto_1 \passp(\mperm,\mpermmap) \\
  \widehat{\OK}^*(\mktab,\mperm,\kconsm{\mkframe}{\mpermmap}{\mctx}) &\stepto_2 b \text{ where }\\ &\phantom{\stepto_2} b \in \setbuild{\passp(\mperm,\mpermmap) \textbf{ and } b}{
          \makont \in \mktab(\mctx),
          \\&\phantom{\stepto_2 b \in \{}b \in \widehat{\OK}(\mktab,\mperm\setminus\mpermmap^{-1}(\Grant),\makont))}
\end{align*}
This definition works fine with the reentrant $\terminal$ function with a dynamically bound \emph{seen} set, but otherwise needs to be rewritten to accumulate the boolean result as a parameter of $\widehat{OK}^*$.
%
We use the accumulator version in the proof for simplicity.

\begin{lemma}[Correctness of $\widehat{\OK}$]
  For all $\mktab,\mperm,\mkont,\makont$,
  \begin{itemize}
  \item{\textbf{Soundness:} if $\mkont \in \unroll{\mktab}{\makont}$ then $\OK(\mperm,\mkont) \in \widehat{\OK}(\mktab,\mperm,\makont)$.}
  \item{\textbf{Completeness:} if $b \in \widehat{\OK}(\mktab,\mperm,\makont)$ then there is a $\mkont \in \unroll{\mktab}{\makont}$ such that $b = \OK(\mperm,\mkont)$.}
  \end{itemize}
\end{lemma}
With this lemma in hand, the correctness proof is almost identical to the core proof of correctness.
\begin{theorem}[Correctness]
  The abstract semantics is sound and complete in the same sense as \autoref{thm:pushdown-correct}.
\end{theorem}
\section{Relaxing contexts for delimited continuations}\label{sec:delim}
What happens when we apply the techniques of the previous sections to a semantics that treats continuations as first-class?
%
The glaring issue is that continuations become ``storeable'' and relevant to the execution of functions.
%
But, it was precisely the \emph{irrelevance} that allowed the separation of $\mstore$ and $\mktab$.
%
Specifically, the store components of continuations become elements of the store's codomain --- a recursion that can lead to an unbounded state space and therefore a non-terminating analysis.
%
We apply the AAM methodology to cut out the recursion; whenever a continuation is captured to go into the store, we allocate an address to approximate the store component of the continuation.
%%

%%
Where should these addresses be mapped?
%
We introduce a new environment, $\mmktab$, that maps these addresses to the stores they represent.
%
The stores that contain addresses in $\mmktab$ are then \emph{open}, and must be paired with $\mmktab$ to be \emph{closed}.
%
This poses the same problem as before with contexts in storeable continuations.
%
Therefore, we give up some precision to regain termination by \emph{flattening} these environments when we capture continuations.
%
Fresh allocation still maintains the concrete semantics, but we necessarily lose some ability to distinguish contexts in the abstract.
%%

\subsection{Case study: shift and reset}
%%
Our concrete test subject is the abstract machine for shift and reset, adapted from \citet{ianjohnson:Biernacki2006274} in the ``{\bf ev}al, {\bf co}ntinue'' style in \autoref{fig:shift-reset}.
%
The figure elides the rules for standard function calls.
%
The new additions to the state space are a new kind of value, $\vcomp{\mkont}$, and a \emph{meta-continuation}, $\mmkont \in \MKont = \Kont^*$ for separating continuations by their different prompts.
%
Composable continuations are indistinguishable from functions, so even though the meta-continuation is concretely a list of continuations, its conses are notated as function composition: $\mkapp{\mkont}{\mmkont}$.

\begin{figure}
  \centering
  $\mstate \stepto \mstate'$ \\
  \begin{tabular}{r|l}%{r|ll}
    \hline
% Reset
    $\ev{\sreset{\mexpr}, \menv, \mkont, \mmkont}$
    &
    $\ev{\mexpr, \menv, \epsilon, \mkapp{\mkont}{\mmkont}}$
%    & \textsc{[push prompt]}
    \\
% Pop prompt
    $\co{\epsilon, \mkapp{\mkont}{\mmkont}, \mval}$
    &
    $\co{\mkont, \mmkont, \mval}$
%    & \textsc{[pop prompt]}
    \\
% Shift
    $\ev{\sshift{\mvar}{\mexpr}, \menv, \mkont, \mmkont}$
    &
    $\ev{\mexpr, \extm{\menv}{\mvar}{\vcomp{\mkont}},\epsilon,\mmkont}$
%    & \textsc{[capture continuation]}
    \\
% continuation call
    $\co{\kcons{\kfn{\vcomp{\mkont'}}}{\mkont}, \mmkont, \mval}$
    &
    $\co{\mkont', \mkapp{\mkont}{\mmkont}, \mval}$
%    & \textsc{[compose continuation]}
  \end{tabular}  
  \caption{Machine semantics for shift/reset}
  \label{fig:shift-reset}
\end{figure}
%%

\subsection{Reformulated with continuation stores}
%
The machine in \autoref{fig:shift-reset} is transformed now to have three new tables: one for continuations, one as discussed in the section beginning to close stored continuations, and one for meta-continuations.
%
The first is like previous sections, albeit continuations may now have the approximate form that is storable.
%
The meta-continuation table is much like the continuation table of previous sections because they are not storable.
%
Meta-continuations do not have simple syntactic strategies for bounding their size, so we choose to bound them to size 0.
%
They could be paired with lists of $\sa{Kont}$ bounded at an arbitrary $n \in \nat$, but we simplify for presentation.

Contexts for continuations are still at function application, but now contain the $\mmktab$.
%
Contexts for meta-continuations are in two places: manual prompt introduction via {\tt reset}, or via continuation invocation.
%
At continuation capture time, continuation contexts are approximated to remove $\mstore$ and $\mmktab$ components.
%
The different context spaces are thus:
\begin{align*}
  \msctx \in \ExactContext &::= \tpl{\maval,\maval,\mstore,\mmktab,\mtime} \\
  \mactx \in \sa{Context} &::= \tpl{\maval,\maval,\maddr,\mtime} \\
  \mctx \in \Context &::= \mactx \alt \msctx \\
  \mmctx \in \MContext &::= \tpl{\mexpr,\maenv,\mstore,\mmktab,\mtime}
                       \alt \tpl{\mvkont, \mval, \mstore, \mmktab, \mtime} \\
\end{align*}
%


\begin{figure}
  \centering
  \begin{align*}
    \mastate \in \sa{SR} &::= \ev{\mexpr,\maenv,\makont,\mamkont} \alt \co{\makont,\mamkont,\maval} \\
    s \in \Prestate &::= \mastate,\mstore,\mmktab,\mtime \\
    \State &::= s,\mktab_{\makont},\mktab_{\mamkont} \\
    \mmktab \in \MKTab &= \Addr \finto \wp(\Store) \\
    \makont \in \sa{Kont} &::= \epsilon \alt \kcons{\mkframe}{\mctx} \alt \mctx \\
    \mamkont \in \sa{MKont} &::= \epsilon \alt \mmctx \\
    \mvkont \in \VKont &::= \epsilon \alt \mactx \\
    \mktab_{\makont} \in \KStore &= \ExactContext \finto \wp(\sa{Kont}) \\
    \mktab_{\mamkont} \in \MKStore &= \MContext \finto \wp(\sa{Kont} \times \sa{MKont}) \\
    \maval \in \sa{Storeable} &= \wp(\VKont + (\Value \times \sa{Env})) \\
  \end{align*}
  \caption{Shift/reset abstract semantic spaces}
  \label{fig:shiftreset-spaces}
\end{figure}
%
The approximation and flattening happens in $\approximate$:
\begin{equation*}
  \approximate : \MKTab \times \Addr \times \SKont \to \MKTab \times \VKont
\end{equation*}
\begin{align*}
  \approximate(\mmktab,\maddr,\epsilon) &= \mmktab,\epsilon \\
  \approximate(\mmktab,\maddr,\kcons{\mkframe}{\tpl{\maval_f,\maval_a,\mstore,\mmktab',\mtime}}) &= \joinm{\mmktab\sqcup\mmktab'}{\maddr}{\mstore},\kcons{\mkframe}{\tpl{\maval_f,\maval_a,\maddr,\mtime}} \\
  \approximate(\mmktab,\maddr,\kcons{\mkframe}{\tpl{\maval_f,\maval_a,\maddralt,\mtime}}) &= \joinm{\mmktab}{\maddr}{\mmktab(\maddralt)},\kcons{\mkframe}{\tpl{\maval_f,\maval_a,\maddr,\mtime}}
\end{align*}
The second case is where continuation closures get flattened together.
%
The third case is when an already approximate continuation is approximated: the approximation is inherited.
%
Approximating the context and allocating the continuation in the store require two addresses, so we relax the specification of $\alloc$ to allow multiple address allocations in this case.

Each of the four rules of the original shift/reset machine has a corresponding rule that we explain piecemeal.
%
We will write $\mastate \kindastepto \mastate',\ldots$ optionally updating the other components instead of the full $\mastate,\mstore,\ldots \stepto \ldots$ to lessen the notational noise.
%
We use the above $\approximate$ function in the rule for continuation capture, as modified here.
%
\begin{equation*}\ev{\sshift{\mvar}{\mexpr},\maenv,\makont,\mamkont} \kindastepto
  \ev{\mexpr,\maenv',\epsilon,\mamkont},\mstore',\mmktab'
\end{equation*}
where
\begin{align*}
  (\maddr,\maddr') &= \alloc(\mctx) & \maenv' &= \extm{\maenv}{\mvar}{\maddr} \\
  (\mvkont,\mmktab') &= \approximate(\mmktab,\maddr',\makont) &
  \mstore' &= \joinm{\mstore}{\maddr}{\mvkont}
\end{align*}

The rule for {\tt reset} stores the continuation and meta-continuation in $\mktab_{\mamkont}$:
\begin{align*}
\ev{\sreset{\mexpr},\maenv,\makont,\mamkont} &\kindastepto
  \ev{\mexpr,\maenv,\epsilon,\mmctx},\joinm{\mktab_{\mamkont}}{\mmctx}{(\makont,\mamkont)} \\
\text{where } \mmctx &= \tpl{\mexpr,\maenv,\mstore,\mmktab,\mtime}
\end{align*}

The prompt-popping rule simply dereferences $\mktab_{\mamkont}$:
\begin{align*}
  \co{\epsilon,\mmctx,\maval} &\kindastepto \co{\makont,\mamkont,\maval} \text{ if } (\makont,\mamkont) \in \mktab_{\mamkont}(\mmctx)
\end{align*}

The continuation installation rule extends $\mktab_{\mamkont}$ at the different context:
\begin{align*}
  \co{\makont,\mamkont,\maval} &\kindastepto \co{\mvkont,\mmctx,\maval},\joinm{\mktab_{\mamkont}}{\mmctx}{(\makont',\mamkont)} \\ 
\text{if } & (\appr{\mvkont},\makont') \in \pop(\mktab_{\makont},\mmktab, \makont) \\
\text{where } \mmctx &= \mvkont,\maval,\mstore,\mmktab,\mtime
\end{align*}
Again we have a metafunction $\pop$, but this time to interpret approximated continuations:
\begin{align*}
  \pop(\mktab_{\makont}, \mmktab, \makont) &= \popaux(\makont,\emptyset) \\
  \text{where } 
   \popaux(\epsilon, G) &= \emptyset \\
   \popaux(\kcons{\mkframe}{\mctx}, G) &= \set{(\mkframe,\mctx)} \\
   \popaux(\mctx, G) &= \bigcup\limits_{\makont \in G'}(\popaux(\makont, G\cup G')) \\
    \text{where } G' &= \bigcup\limits_{\msctx \in I(\mctx)}{\mktab_{\makont}(\msctx)} \setminus G \\
  I(\msctx) &= \set{\msctx} \\
  I(\tpl{\maval_f,\maval_a,\maddr,\mtime}) &=
  \setbuild{\tpl{\maval_f,\maval_a,\mstore,\mmktab',\mtime} \in \dom(\mktab_{\makont})}
           {\mstore \in \mmktab(\maddr),
            \mmktab' \sqsubseteq \mmktab}
\end{align*}
Notice that since we flatten $\mmktab$s together, we need to compare for containment rather than for equality (in $I$).
\subsection{Correctness}
We suppose a store-allocating timestamped version of the semantics in \autoref{fig:shift-reset}, called $\SRSt$.
%
We impose an order on values since stored continuations are more approximate in the analysis than in $\SRSt$:
\begin{mathpar}
  \inferrule{ }{\maval \sqsubseteq_{\mktab,\mmktab} \maval} \quad
  \inferrule{\mkont \in \unroll{\mktab,\mmktab}{\mvkont}}
            {\vcomp{\mkont} \sqsubseteq_{\mktab,\mmktab} \mvkont} \quad
  \inferrule{\forall \maval\in\mstore(\maddr).
             \exists \maval'\in\mastore(\maddr).
             \maval \sqsubseteq_{\mktab,\mmktab} \maval'}
            {\mstore \sqsubseteq_{\mktab,\mmktab} \mastore} \\
  \inferrule{\mkont \sqsubseteq \unroll{\mktab_{\makont},\mmktab}{\makont} \\
             \mmkont \sqsubseteq \unrollC{\mktab_{\makont},\mktab_{\mamkont},\mmktab}{\mamkont} \\
             \mstore \sqsubseteq_{\mktab_{\makont},\mmktab} \mastore}
            {\ev{\mexpr,\maenv,\mkont,\mmkont},\mstore,\mtime \sqsubseteq
             \ev{\mexpr,\maenv,\makont,\mamkont},\mastore, \mmktab, \mtime, \mktab_{\makont}, \mktab_{\mamkont}} \\
  \inferrule{\maval \sqsubseteq_{\mktab_{\makont},\mmktab} \maval' \\
             \mkont \sqsubseteq \unroll{\mktab_{\makont},\mmktab}{\makont} \\
             \mmkont \sqsubseteq \unrollC{\mktab_{\makont},\mktab_{\mamkont},\mmktab}{\mamkont} \\
             \mstore \sqsubseteq_{\mktab_{\makont},\mmktab} \mastore}
            {\co{\mkont,\mmkont,\maval},\mstore,\mtime \sqsubseteq
             \co{\makont,\mamkont,\maval'},\mastore, \mmktab, \mtime, \mktab_{\makont}, \mktab_{\mamkont}}
\end{mathpar}
Unrolling differs from the previous sections because the values in frames can be approximate.
%
Thus, instead of expecting the exact continuation to be in the unrolling, we have a judgment that an unrolling approximates a given continuation in \autoref{fig:cont-order} (note we reuse $I$ from $\popaux$'s definition).

\begin{figure}
  \centering
  \begin{mathpar}
    \inferrule{ }{\appl{\mexpr,\maenv} \sqsubseteq_{\mktab,\mmktab}
      \appl{\mexpr,\maenv}} \quad \inferrule{\maval
      \sqsubseteq_{\mktab,\mmktab}{\maval'}}
    {\appr{\maval} \sqsubseteq_{\mktab,\mmktab} \appr{\maval'}} \\
    \inferrule{ }{\epsilon \sqsubseteq
      \unroll{\mktab,\mmktab}{\epsilon}} \quad
    \inferrule{\mkframe \sqsubseteq_{\mktab,\mmktab} \makframe \\
      \mkont \sqsubseteq \unroll{\mktab,\mmktab}{\mctx}}
    {\kcons{\mkframe}{\mkont} \sqsubseteq
      \unroll{\mktab,\mmktab}{\kcons{\makframe}{\mctx}}}
    \\
    \inferrule{\makont \in \mktab(\msctx) \quad
      \mkont \sqsubseteq \unroll{\mktab,\mmktab}{\makont}} {\mkont
      \sqsubseteq \unroll{\mktab,\mmktab}{\msctx}}
    \quad
    \inferrule{\msctx \in I(\mktab,\mmktab,\mactx) \quad
      \mkont \sqsubseteq \unroll{\mktab,\mmktab}{\msctx}} {\mkont
      \sqsubseteq \unroll{\mktab,\mmktab}{\mactx}}
    \\
    \inferrule{ }
              {\epsilon \sqsubseteq \unrollC{\mktab_{\makont},\mktab_{\mamkont},\mmktab}{\epsilon}}
    \\
    \inferrule{(\makont,\mamkont) \in \mktab_{\mamkont}(\mmctx) \\
               \mkont \sqsubseteq \unroll{\mktab_{\makont},\mmktab}{\makont} \\
               \mmkont \sqsubseteq \unrollC{\mktab_{\makont},\mktab_{\mamkont},\mmktab}{\mamkont}}
              {\mkapp{\mkont}{\mmkont} \sqsubseteq \unrollC{\mktab_{\makont},\mktab_{\mamkont},\mmktab}{\mmctx}}
  \end{mathpar}
  
  \caption{Order on (meta-)continuations}
\label{fig:cont-order}
\end{figure}
\begin{theorem}[Soundness]
  If $\mstate,\mstore,\mtime \stepto_{\SRSt} \mstate',\mstore',\mtime'$, and $\mstate,\mstore,\mtime$ $\sqsubseteq$ $\mastate,\mastore,\mmktab,\mtime,\mktab_{\makont},\mktab_{\mamkont}$ then there are $\mastate',\mastore',\mmktab',\mktab_{\makont}',\mktab_{\mamkont}'$ such that $\mastate,\mastore,\mmktab,\mtime,\mktab_{\makont},\mktab_{\mamkont} \stepto \mastate',\mastore',\mmktab',\mtime,\mktab_{\makont}',\mktab_{\mamkont}'$ and
$\mstate',\mstore',\mtime'$ $\sqsubseteq$ $\mastate',\mastore',\mmktab',\mtime',\mktab_{\makont}',\mktab_{\mamkont}'$.
\end{theorem}

To substantiate our claim that fresh allocation leads to an exact semantics, we need to show an invariant on our continuation stores.
\begin{align*}
  \inv(\mastate,\mastore,\mmktab,\mtime,\mktab_{\makont},\mktab_{\mamkont})
\end{align*}
\begin{theorem}[Exact for fresh allocation]
  \todo{Ian}
\end{theorem}

\section{Short-circuiting via ``summarization''}\label{sec:memo}
All the semantics of previous sections have a performance weakness that many analyses share: unnecessary propagation.
%
Consider two portions of a program that do not affect one another's behavior.
%
Both can change the store, and the semantics will be unaware that the changes will not interfere with the other's execution.
%
The more possible stores there are in execution, the more possible contexts in which a function will be evaluated.
%
Multiple independent portions of a program may be reused with the same arguments and store contents they depend on, but changes to irrelevant parts of the store lead to redundant computation.
%
The idea of skipping from a change past several otherwise unchanged states to uses of the change is called ``sparseness'' in the literature~\citep{ianjohnson:Reif1977Symbolic,ianjohnson:Wegman1991Constant,ianjohnson:DBLP:conf/pldi/OhHLLY12}.
%

%
Memoization is a specialized instance of sparseness; the base stack may change, but the evaluation of the function does not, so given an already computed result we can jump straight to the answer.
%
We use the vocabulary of ``relevance'' and ``irrelevance'' so that future work can adopt the ideas of sparseness to reuse contexts in more ways.
%

Recall the \hyperref[lem:irrelevance]{context irrelevance lemma}: if we have seen the results of a computation before from a different context, we can reuse them.
%
The semantic counterpart to this idea is a memo table that we extend when popping and appeal to when about to push.
%
This simple idea works well with a deterministic semantics, but the non-determinism of abstraction requires care.
%
In particular, memo table entries can end up mapping to multiple results, but not all results will be found at the same time.
%
Note the memo table space:
\begin{align*}
  \mmemo \in \Memo &= \Context \finto \wp(\Relevant) \\
  \Relevant &::= \tpl{\mexpr,\menv,\mstore}
\end{align*}
%
There are a few ways to deal with multiple results:
\begin{enumerate}
\item{rerun the analysis with the last memo table until the table doesn't change (expensive),}
\item{short-circuit to the answer but also continue evaluating anyway (negates most benefit of short-circuiting), or}
\item{use a frontier-based semantics like in \autoref{sec:eng-frontier} with global $\mktab$ and $\mmemo$, taking care to
    \begin{enumerate}
    \item{at memo-use time, still extend $\mktab$ so later memo table extensions will ``flow'' to previous memo table uses, and}
    \item{when $\mktab$ and $\mmemo$ are extended at the same context at the same time, also create states that act like the $\mmemo$ extension point also returned to the new continuations stored in $\mktab$.}
    \end{enumerate}}
\end{enumerate}

We will only discuss the final approach.
%
The same result can be achieved with a one-state-at-a-time frontier semantics, but we believe this is cleaner and more parallelizable.
%
Its second sub-point we will call the ``push/pop rendezvous.''
%
The rendezvous is necessary because there may be no later push or pop steps that would regularly appeal to either (then extended) table at the same context.
%
The frontier-based semantics then makes sure these pushes and pops find each other to continue on evaluating.
%
In pushdown and nested word automata literature, the push to pop short-circuiting step is called a ``summary edge'' or with respect to the entire technique, ``summarization.''
%
We find the memoization analogy appeals to programmers' and semanticists' operational intuitions.
%

%
A second concern for using memo tables is soundness.
%
Without the completeness property of the semantics, memoized results in, \eg{}, an inexactly GC'd machine, can have dangling addresses since the possible stacks may have grown to include addresses that were previously garbage.
%
These addresses would not be garbage at first, since they must be mapped in the store for the contexts to coincide, but during the function evaluation the addresses can become garbage.
%
If they are supposed to then be live, and are used (presumably they are reallocated post-collection), the analysis will miss paths it must explore for soundness.
%

\iftr{
Context-irrelevance is a property of the semantics \emph{without} continuation stores, so there is an additional invariant to that of \autoref{sec:pushdown} for the semantics with $\mktab$ and $\mmemo$: $\mmemo$ respects context irrelevance.
\begin{mathpar}
  \inferrule{\dom(\mmemo) \subseteq \dom(\mktab) \\
             \forall \mctx\equiv\tpl{\mexpr_c,\menv_c,\mstore_c} \in \dom(\mmemo),
                     \tpl{\mexpr_r,\menv_r,\mstore_r} \in \mmemo(\mctx), \\
                     \makont\in\mktab(\mctx),
                     \mkont\in\unroll{\mktab}{\makont}. \\
              \exists\mtrace\equiv\tpl{\mexpr_c,\menv_c,\mstore_c,\mkont} \stepto_{\CESKt}^* \tpl{\mexpr_r,\menv_r,\mstore_r,\mkont}. \hastail(\mtrace,\mkont)}
            {\inv_M(\mktab,\mmemo)}
\end{mathpar}
Inexact GC does \emph{not} respect context irrelevance for the same reasons it is not complete: some states are spurious, meaning some memo table entries will be spurious, and the expected path in the invariant will not exist.
%
The reason we use unrolled continuations instead of simply $\epsilon$ for this (balanced) path is precisely for stack inspection reasons.
}

 \begin{figure}
   \begin{center}
     $\mastate,\mktab,\mmemo \stepto
     \mastate',\mktab',\mmemo'$
     \begin{tabular}{r|l}
       \hline\vspace{-3mm}\\
       $\tpl{\sapp{\mexpri0}{\mexpri1},\menv,\mstore,\makont},\mktab,\mmemo$
       &
       $\tpl{\mexpri0,\menv,\mstore,\kcons{\appl{\mexpri1,\menv}}{\mctx}},\mktab,\mmemo$ \\
       & \quad if $\mctx \notin\dom(\mmemo)$, or \\
       &
       $\tpl{\mexpr',\menv',\mstore',\makont},\mktab',\mmemo$ \\
       & \quad if $\tpl{\mexpr',\menv',\mstore'} \in \mmemo(\mctx)$ \\
       where & $\mctx = \tpl{\sapp{\mexpri0}{\mexpri1},\menv,\mstore}$ \\
       & $\mktab' = \joinm{\mktab}{\mctx}{\makont}$
       \\
       $\tpl{\mval,\mstore,\kcons{\appr{\slam{\mvar}{\mexpr},\menv}}{\mctx}},\mktab,\mmemo$
       &
       $\tpl{\mexpr,\menv',\mstore',\makont},\mktab,\mmemo'$ if $\makont \in \mktab(\mctx)$ \\
       where & $\menv' = \extm{\menv}{\mvar}{\maddr}$ \\
       & $\mstore' = \joinm{\mstore}{\maddr}{\mval}$ \\
       & $\mmemo' = \joinm{\mmemo}{\mctx}{\tpl{\mexpr,\menv',\mstore'}}$
     \end{tabular}
   \end{center}
   \caption{Important memoization rules}
   \label{fig:memo}
 \end{figure}

The rules in \autoref{fig:memo} are the importantly changed rules from \autoref{sec:pushdown} that short-circuit to memoized results.
%
The technique looks more like memoization with a $\CESIKKstart$ machine, since the memoization points are truly at function call and return boundaries.
%
The $\pop$ function would need to also update $\mmemo$ if it dereferences through a context, but otherwise the semantics are updated \emph{mutatis mutandis}.

\begin{equation*}
  {\mathcal F}_{\mexpr}(S,R,F,\mktab,\mmemo) = (S \cup F, R \cup R', F'\setminus S, \mktab', \mmemo')
\end{equation*}
where

\begin{tabular}{rlrlrl}
  $I$ &
  \multicolumn{5}{l}{
    \hspace{-3mm}$=\bigcup\limits_{\mstate \in
      F}{\setbuild{(\tpl{\mstate,\mstate'}, \mktab',\mmemo')}{\mstate,\mktab,\mmemo
        \stepto \mstate',\mktab',\mmemo'}}$}
\\
   $R'$ &\hspace{-3mm}$= \pi_0 I$ & $\mktab'$ & \hspace{-3mm}$= \bigsqcup\pi_1 I$ & $\mmemo'$ & \hspace{-3mm}$= \bigsqcup\pi_2 I$ \\
   $\Delta\mktab$ &\hspace{-3mm}$= \mktab'\setminus\mktab$ & $\Delta\mmemo$ & \hspace{-3mm}$= \mmemo'\setminus\mmemo$ & & \\
   $F'$ &
   \multicolumn{5}{l}{
     \hspace{-3mm}$= \pi_1 R' \cup \{{\tpl{\mexpr,\menv,\mstore,\makont}} :
     {\mctx \in \dom(\Delta\mktab)\cap\dom(\Delta\mmemo).}$}
   \\ &\multicolumn{5}{l}{\hspace{-3mm}$\phantom{= \pi_1 R' \cup \{} \makont \in \Delta\mktab(\mctx),
       \tpl{\mexpr,\menv,\mstore} \in \Delta\mmemo(\mctx)\}$}
 \end{tabular}

\iftr{
\begin{theorem}[Correctness]
Same property is the same as in \autoref{thm:global-pushdown}, where $\reify$ ignores the $\mmemo$ component.
\end{theorem}
The proof appeals to the invariant on $\mmemo$ whose proof involves an additional argument for the short-circuiting step that reconstructs the path from a memoized result using both context irrelevance and the table invariants.
}
%  LocalWords:  parallelizable pushdown automata summarization
%  LocalWords:  memoization memoized dereferences

\section{Related Work}
%%
The immediately related work is that of PDCFA \citep{dvanhorn:Earl2010Pushdown, dvanhorn:Earl2012Introspective}, CFA2~\citep{dvanhorn:Vardoulakis2011CFA2, dvanhorn:Vardoulakis2011Pushdown}, and AAM~\citep{dvanhorn:VanHorn2010Abstracting}.
%
The stack frames in CFA2 that boost precision are an orthogonal feature that fit into our model as an \emph{irrelevant} component along with the stack, which we did not cover in detail due to space constraints.
%
The version of CFA2 that handles \rackett{call/cc} does not handle composable control, is dependent on a restricted CPS representation, and has untunable precision for first-class continuations.
%
Our semantics adapts to \rackett{call/cc} by removing the meta-continuation operations, and thus this work supersedes theirs; the machinery is in fact a strict generalization.
%
The extended version of PDCFA that inspects the stack to do garbage collection~\citep{dvanhorn:Earl2012Introspective} also fits into our model (\autoref{sec:gc}'s $\hat\Gamma$).
%
We suspect the more ``semantic'' garbage collection from \citet{mc-via-gamma} can be easily adapted to the pushdown setting.

We did additional work to improve the performance of the AAM approach in \citet{dvanhorn:Johnson2013Optimizing} that can almost entirely be imported for the work in this paper.
%
The technique that does not apply is ``store counting'' for lean representation of the store component of states when there is a global abstract store, an assumption that does not hold for garbage collection.
%
The implementation\footnote{\url{http://github.com/dvanhorn/oaam}} has pushdown modules that use the ideas in this paper.

%%
\paragraph{Stack inspection}
Stack inspecting flow analyses also exist, but operate on pre-constructed regular control-flow graphs~\citep{ianjohnson:bartoletti2004stack}, so the CFGs cannot be trimmed due to the extra information at construction time, leading to less precision.
%
Backward analyses for stack inspection also exist, with the same prerequisite~\citep{ianjohnson:DBLP:journals/sigplan/Chang06}.
%%

\paragraph{Pushdown models and memoization}
The idea of relating pushdown automata with memoization is not new.
%
In 1971, Stephen Cook~\citep{DBLP:conf/ifip/Cook71} devised a transformation to simulate 2-way (on a fixed input) \emph{deterministic} pushdown automata in time linear in the size of the input, that uses the same ``context irrelevance'' idea to skip from state $q$ seen before to a corresponding first state that pops the stack below where $q$ started (called a \emph{terminator} state).
%
Six years later, Neil D. Jones~\citep{Jones:1977:NLT} simplified the transformation instead to \emph{interpret} a stack machine program to work \emph{on-the-fly} still on a deterministic machine, but with the same idea of using memo tables to remember corresponding terminator states.
%
Thirty-six years after that, at David Schmidt's Festschrift, Robert Gl\"uck extended the technique to 2-way \emph{non-deterministic} pushdown automata, and showed that the technique can be used to recognize context-free languages in the standard ${\mathcal O}(n^3)$ time~\citep{DBLP:journals/corr/Gluck13}.
%
Gl\"uck's technique (as yet, correctness unproven) uses the meta-language's stack with a deeply recursive interpretation function to preclude the use of a frontier and something akin to $\mktab$\footnote{See \texttt{gluck.rkt} in online materials for Gl\"uck's style}.
%
By exploring the state space \emph{depth-first}, the interpreter can find all the different terminators a state can reach one-by-one by destructively updating the memo table with the ``latest'' terminator found.
%
The trade-offs with this technique are that it does not obviously scale to first-class control, and the search can overflow the stack when interpreting moderate-sized programs.
%
A minor disadvantage is that it is also not a fair evaluation strategy when allocation is unbounded.
%
The technique can nevertheless be a viable alternative for languages with simple control-flow mechanisms.
%
It has close similarities to ``Big-CFA2'' in Vardoulakis' dissertation~\citep{vardoulakis-diss12}.
%%

%%
\paragraph{Analysis of pushdown automata}
Pushdown models have existed in the first-order static analysis literature~\citep[Chapter 7]{local:muchnick:jones:flow-analysis:1981}\citep{dvanhorn:Reps1995Precise}, and the first-order model checking literature \citep{dvanhorn:Bouajjani1997Reachability}, for some time.
%
These methods already assume a pushdown model as input, and constructing a model from a first-order program is trivial.
%
In the setting of higher-order functions and first-class control, model construction is an additional problem -- the one we solve here.
%
Additionally, the algorithms employed in these works expect a complete description of the model up front, rather than work with a modified \texttt{step} function (also called \texttt{post}), such as in ``on-the-fly'' model-checking algorithms for finite state systems~\citep{dvanhorn:Schwoon2005Note}.
%%

%%
\paragraph{Derivation from abstract machines}
The trend of deriving static analyses from abstract machines does not stop at flow analyses.
%
The model-checking community showed how to check temporal logic queries for collapsible pushdown automata (CPDA), or equivalently, higher-order recursion schemes, by deriving the checking algorithm from the Krivine machine~\citep{dvanhorn:Salvati2011Krivine}.
%
The expressiveness of CPDAs outweighs that of PDAs, but it is unclear how to adapt higher-order recursion schemes (HORS) to model arbitrary programming language features.
%
The method is strongly tied to the simply-typed call-by-name lambda calculus and depends on finite sized base-types.
%
%HORSs are a powerful abstraction target for which monadic second order logic (MSO) is decidable, but systematic constructions of models in HORS from arbitrary programming languages are still out of reach.

\section{Conclusion}
As the programming world continues to embrace behavioral values, it becomes more important to import the powerful techniques pioneered by the first-order analysis and model-checking communities.
%
CFA2 and PDCFA paved the way, and in large part inspired this work.
%
We make a modest attempt to explain the core of their techniques in a common framework and extend it in a novel and instructive way.
%
It is our view that systematic approaches to applying the techniques are pivotal to scaling them to production programming languages.
%
We showed how to systematically construct executable concrete semantics that point-wise abstract to known algorithms for pushdown analyses of higher-order languages.
%
We bypass the automata theoretic approach so that we are not chained to a pushdown automaton to model features such as first-class composable control operators.
%
The techniques still apply and give better precision than regular methods.

\balance
\bibliographystyle{plainnat}
\bibliography{bibliography}

\end{document}