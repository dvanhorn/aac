%%
The immediately related work is that of PDCFA \citep{dvanhorn:Earl2010Pushdown, dvanhorn:Earl2012Introspective}, CFA2~\citep{dvanhorn:Vardoulakis2011CFA2, dvanhorn:Vardoulakis2011Pushdown}, and AAM~\citep{dvanhorn:VanHorn2010Abstracting}.
%
The stack frames in CFA2 that boost precision are an orthogonal feature that fit into our model as an \emph{irrelevant} component along with the stack, which we did not cover in detail due to space constraints.
%
The version of CFA2 that handles \rackett{call/cc} does not handle composable control, is dependent on a restricted CPS representation, and has untunable precision for first-class continuations.
%
Our semantics adapts to \rackett{call/cc} by removing the meta-continuation operations, and thus this work supersedes theirs; the machinery is in fact a strict generalization.
%
The extended version of PDCFA that inspects the stack to do garbage collection~\citep{dvanhorn:Earl2012Introspective} also fits into our model (\autoref{sec:gc}'s $\hat\Gamma$).
%
We suspect the more ``semantic'' garbage collection from \citet{mc-via-gamma} can be easily adapted to the pushdown setting.

We did additional work to improve the performance of the AAM approach in \citet{dvanhorn:Johnson2013Optimizing} that can almost entirely be imported for the work in this paper.
%
The technique that does not apply is ``store counting'' for lean representation of the store component of states when there is a global abstract store, an assumption that does not hold for garbage collection.
%
The implementation\footnote{\url{http://github.com/dvanhorn/oaam}} has pushdown modules that use the ideas in this paper.

%%
\paragraph{Stack inspection}
Stack inspecting flow analyses also exist, but operate on pre-constructed regular control-flow graphs~\citep{ianjohnson:bartoletti2004stack}, so the CFGs cannot be trimmed due to the extra information at construction time, leading to less precision.
%
Backward analyses for stack inspection also exist, with the same prerequisite~\citep{ianjohnson:DBLP:journals/sigplan/Chang06}.
%%

\paragraph{Pushdown models and memoization}
The idea of relating pushdown automata with memoization is not new.
%
In 1971, Stephen Cook~\citep{DBLP:conf/ifip/Cook71} devised a transformation to simulate 2-way (on a fixed input) \emph{deterministic} pushdown automata in time linear in the size of the input, that uses the same ``context irrelevance'' idea to skip from state $q$ seen before to a corresponding first state that pops the stack below where $q$ started (called a \emph{terminator} state).
%
Six years later, Neil D. Jones~\citep{Jones:1977:NLT} simplified the transformation instead to \emph{interpret} a stack machine program to work \emph{on-the-fly} still on a deterministic machine, but with the same idea of using memo tables to remember corresponding terminator states.
%
Thirty-six years after that, at David Schmidt's Festschrift, Robert Gl\"uck extended the technique to 2-way \emph{non-deterministic} pushdown automata, and showed that the technique can be used to recognize context-free languages in the standard ${\mathcal O}(n^3)$ time~\citep{DBLP:journals/corr/Gluck13}.
%
Gl\"uck's technique (as yet, correctness unproven) uses the meta-language's stack with a deeply recursive interpretation function to preclude the use of a frontier and something akin to $\mktab$\footnote{See \texttt{gluck.rkt} in online materials for Gl\"uck's style}.
%
By exploring the state space \emph{depth-first}, the interpreter can find all the different terminators a state can reach one-by-one by destructively updating the memo table with the ``latest'' terminator found.
%
The trade-offs with this technique are that it does not obviously scale to first-class control, and the search can overflow the stack when interpreting moderate-sized programs.
%
A minor disadvantage is that it is also not a fair evaluation strategy when allocation is unbounded.
%
The technique can nevertheless be a viable alternative for languages with simple control-flow mechanisms.
%
It has close similarities to ``Big-CFA2'' in Vardoulakis' dissertation~\citep{vardoulakis-diss12}.
%%

%%
\paragraph{Analysis of pushdown automata}
Pushdown models have existed in the first-order static analysis literature~\citep[Chapter 7]{local:muchnick:jones:flow-analysis:1981}\citep{dvanhorn:Reps1995Precise}, and the first-order model checking literature \citep{dvanhorn:Bouajjani1997Reachability}, for some time.
%
These methods already assume a pushdown model as input, and constructing a model from a first-order program is trivial.
%
In the setting of higher-order functions and first-class control, model construction is an additional problem -- the one we solve here.
%
Additionally, the algorithms employed in these works expect a complete description of the model up front, rather than work with a modified \texttt{step} function (also called \texttt{post}), such as in ``on-the-fly'' model-checking algorithms for finite state systems~\citep{dvanhorn:Schwoon2005Note}.
%%

%%
\paragraph{Derivation from abstract machines}
The trend of deriving static analyses from abstract machines does not stop at flow analyses.
%
The model-checking community showed how to check temporal logic queries for collapsible pushdown automata (CPDA), or equivalently, higher-order recursion schemes, by deriving the checking algorithm from the Krivine machine~\citep{dvanhorn:Salvati2011Krivine}.
%
The expressiveness of CPDAs outweighs that of PDAs, but it is unclear how to adapt higher-order recursion schemes (HORS) to model arbitrary programming language features.
%
The method is strongly tied to the simply-typed call-by-name lambda calculus and depends on finite sized base-types.
%
%HORSs are a powerful abstraction target for which monadic second order logic (MSO) is decidable, but systematic constructions of models in HORS from arbitrary programming languages are still out of reach.
